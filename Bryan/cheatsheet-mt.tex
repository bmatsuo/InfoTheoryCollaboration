\documentclass[10pt]{article}

\usepackage{fullpage,latexsym,amsthm,amsmath,amsfonts,algorithmic,graphicx,subfigure,color,verbatim,epsfig,fancyhdr}
\pagestyle{empty}
% Define new commands.
\DeclareMathOperator*{\E}{\mbox{\text{E}}}

\newcommand{\pref}[1]{{(\ref{#1})}}
\newcommand{\bmat}[1]{{\color{red}{#1}}}

\newtheorem{defn}{Definition}[section]
\newtheorem{thm}{Theorem}[section]
\newtheorem{lemma}{Lemma}[section]
\newtheorem{corollary}{Corollary}[section]
\newtheorem{question}{Question}[section] \newtheorem{proposition}{Proposition}[section] \usepackage{multicol}
\usepackage{savetrees}


% Begin the actual document content here.
\begin{document}
\begin{tiny}
\begin{multicols}{3}

\begin{align*}
\chi &= \text{State space } (X_i \in \chi) \\
X^n &=(X_1,\ldots,X_n) \\
\end{align*}
\section*{Entropy, Relative Entropy and Mutual Information}
Definitions, alternate expressions, and properties.
\begin{align}
    \underbrace{H(X)}_\text{entropy}&= \sum_{x\in X} p(x) \log\ \frac{1}{p(x)}
    = \E_X \log p(x) \label{defn: entropy}\\
H(X)& \geq 0 \\
H_b(X)&= (\log_b a)H_a(X) \\
H(X|Y)& \leq  H(X)\text{ (eq iff $X,Y$ ind)} \label{eq: H cond}\\
H(X^n)& \leq \sum_{i=1}^n H(X_i)\text{ (eq iff all $X_i$ ind)} \label{eq:jointsum}\\
H(X)&\leq  \log |\chi|  \text{ (eq iff iid) }\label{eq:alphabetentropy}\\
H(p)&\text{ is concave in $p$.}\\
\underbrace{D(p||q)}_\text{rel ent} & = \sum_x p(x) \log \frac{p(x)}{q(x)}=\E_p\log \frac{p(X)}{q(X)}\\
\underbrace{I(X;Y)}_\text{mut inf} & = \sum_{x,)} p(x,y) \log \frac{p(x,y)}{p(x)p(y)} = \E_p \log \frac{p(X,Y)}{p(X)p(Y)}\\
I(X;Y)&=H(X)-H(X|Y)\\
&=H(Y)-H(Y|X)\\
&=H(X)+H(Y)-H(X,Y)\\
D(p||q)&\geq 0 \text{ (eq iff $p=q$)}\\
I(X;Y)&=D(p(x,y)||p(x)p(y))\\
	&\geq 0 \text{ (eq iff $p(x,y)=p(x)p(y)$)}
\end{align}
Given $|\chi|=m$ and $u$ is the uniform distribution over $\chi$, we get equation~\ref{eq:chimu}.
\begin{align}
D(p||u)&=\log m - H(p)\label{eq:chimu}\\
D(p||q)&\text{ is convex in the pair }(p,q)
\end{align}
Following are chain rules:
\begin{align}
H(X^n)&=\sum_{i=1}^n H(X_i|X^{i-1})\\
I(X^n;Y)&=\sum_{i=1}^n I(X_i;Y|X^{i-1})\\
D(p(x,y)||q(x,y))&=D(p(x)||q(x))\\
&+D(p(y|x)||q(y|x))
\end{align}
\thm[Jensen's Inequality] \label{thm: jensen}
if $f$ is a convex function, then
\begin{equation}
Ef(x) \geq f(EX)
\end{equation}
\thm[Log sum inequality] \label{thm: log sum}
for n positive numbers $a_1,\ldots,a_n$ and $b_1,\ldots,b_n$,
\begin{equation}
\sum_{i=1}^n a_i \log \frac{a_i}{b_i} \geq \left( \sum_{i=1}^n a_i \right) \log \frac{\sum_{i=1}^n a_i}{\sum_{i=1}^n b_i}
\end{equation}
with equality only if $\frac{a_i}{b_i}$ is constant. 

\thm[Data processing inequality] \label{thm: data processing} 
given some Markov processes $X\rightarrow Y\rightarrow Z$ we know
\( I(X;Y) \geq I(X;Z) \)

\thm[Sufficient Statistics] \label{thm: suff stats}
$T(X)$ is sufficient relative to $\{f_\theta (x)\}$ iff $I(\theta;X)=I(\theta;T(X))$ for all distributions on $\theta$.

\thm[Fano's inequality] \label{thm: fano}
Let $P_e=Pr\{\hat{X}(Y)\neq X\}$. Then $H(P_e)+P_e \log |\chi| \geq H(X|Y)$.

\thm[Inequality] \label{thm: Ineq}
If $X$ and $X'$ are independent and identically distributed then $Pr(X=X') \geq 2^{-H(X)}$


\section*{Asymptotic Equipartition Property (AEP)}
``Almost all events are almost equally surprising''. Specifically if $X_1,\ldots,X_n$ are iid with respect to $p(x)$ then :
\begin{equation}
-\frac{1}{n}\log p(X_1,\ldots,X_n)\rightarrow H(X) \text{in probability}
\end{equation}

A typical set $A_\epsilon^{(n)}$ is the set of sequences $x_1,\ldots, x_n$ that follow:
\begin{equation}
2^{-n(H(X)+\epsilon)}\leq p(X^n)\leq 2^{-n(H(X)-\epsilon)}
\end{equation}

Some properties of a typical set:
\begin{enumerate}
\item if $(x_1,\ldots,x_n) \in A_\epsilon^{(n)}$ then $p(x_1,\ldots,x_n) = 2^{-n(H\pm \epsilon)}$
\item Pr$\left\{A_\epsilon^{(n)}\right\}>1-\epsilon$ for sufficiently large $\epsilon$
\item $|A_\epsilon^{(n)}| \leq 2^{n(H(X)+\epsilon}$
\end{enumerate}

Definition of $a_n \doteq b_n$ means that $\frac{1}{n}\log \frac{a_n}{b_n} \rightarrow 0$ as $n\rightarrow \infty$.

Smallest probable set: Let $X_1,\ldots, X_n$ be iid with respect to $p(x)$ and for $\gamma < \frac{1}{2}$, let $B_\gamma^{(n)} \subset \chi^n$ be the smallest set such that $Pr\{ B_\gamma^{(n)} \} \geq 1-\gamma $ Then we have $|B_\gamma^{(n)}| \doteq 2^{nH}$

\section*{Other useful laws and properties}
{\bf Law of Large Numbers} Weak law:
\begin{eqnarray*}
\bar{X}_n\overset{p}\rightarrow \mu \text{ when } n \rightarrow \infty\\
\lim_{n\rightarrow \infty} Pr\left(|\bar{X}_n-\mu| < \epsilon \right) = 1
\end{eqnarray*}
Strong law:
\begin{equation}
Pr\left(\lim_{n\rightarrow \infty} \bar{X}_n = \mu \right) = 1
\end{equation}

\section*{Entropy Rate}
\begin{eqnarray}
H(\chi) = lim_{n \to \infty} \frac{1}{n} H(X^n) \\
H^\prime(\chi) = lim_{n \to \infty} H(X_n | X^{n-1}) \\
H(\chi) = H^\prime(\chi) \text{ (If stationary and stochastic)}
\end{eqnarray}

\noindent Entropy rate of stationary Markov chain (transition matrix $P$) :
\begin{equation}
H(\chi) = H(X_2|X_1) = -\sum_{i,j} \mu_i P_{ij}log(P_ij)
\end{equation}

\noindent{\bf Stationarity } if a distribution at time $n+1$ is the same as time $n$, it is called a {\it stationary distribution}. If the initial state of Markov chain is a {\it stationary distribution} then the Markov chain is a stationary process. A finite-state Markov chain that is irreducible (positive prob. from any state to any other in finite steps) and aperiodic (largest factor of length of paths of state to itself is 1) has a unique stationary distribution that $lim_{n \to \infty}X_n$ converges to.

\noindent{\bf Functions of Markov Chain}
$X^n$ is stationary Markov chain, and $Y_i = \phi(X_i)$ then

\begin{eqnarray*}
H(Y_n | Y^{n-1}, X_1) \leq H(\gamma) \leq H(Y_n|Y^{n-1})\\
lim_{n \to \infty}H(Y_n|Y^{n-1},X_1) = H(\gamma) \\
= lim_{n \to \infty} H(Y_n | Y^{n-1})
\end{eqnarray*}

\noindent{\bf Ces\'{a}ro mean} If $a_n \to a$ and $b_n = \frac{1}{n}\sum_{i=1}^n a_i$ then $b_n \to a$. 

\noindent{\bf 2 state Markov Chain} given $P=[1-\alpha, \alpha; \beta, 1-\beta]$ the stationary distribution $u$ where $uP = u$, is $u_1 = \frac{\beta}{\alpha+\beta}$, $u_2 = \frac{\alpha}{\alpha+\beta}$, and $H(\chi) = \frac{\beta}{\alpha + \beta}H(\alpha) + \frac{\alpha}{\alpha+\beta}H(\beta)$. 

\noindent{\bf Doubly Stochastic} probability transition matrix $[P_{ij}] = Pr\{X_{n+1} = j | X_n = i\}$ is doubly stochastic if rows and cols sum to $1$. The uniform distribution is a stationary distribution iff $P$ is double stochastic.

% HW1
\section* {2.2 Entropy of functions.}
Let $Y=g(X)$. The distribution for $Y$ is  $p(y)=\sum_{\{x|y=g(x)\}}p(x)$.  It is pretty clear that if there is a 1-1 mapping then $p(x)=p(g(x))$. 
                          
Consider the situation where multiple values of $X$ that map to a single value of $Y$.
For this set $\sum_{\{x|y=g(x)\}}p(x)\ log\ p(x) \leq \sum_{\{x|y=g(x)\}}p(x)\ log\ p(y) = p(y)\ log\ p(y)$ since $p(y)\geq p(x)$. Now extending the argument to the entire range of $X$ and $Y$ we obtain:
\(
p(x)\leq \sum_{\{x|y=g(x)\}}p(x)=p(y)\) and \(
H(X)=-\sum_{x}p(x)\ log\ p(x)
=-\sum_{y}\sum_{\{x|y=g(x)\}}p(x)\ log\ p(x)
\geq -\sum_{y}p(y)\ log\ p(y)
=H(Y)
\)

\textbf{(a)} $H(X)$ vs $H(Y=2^{X})$\\
$Y$ and $X$ are 1-1 so we have the special case of equality.

\textbf{(b)} $H(X)$ vs $H(Y=cos(X))$ \\
There can be multiple values of $X$ that map to single values of $Y$ with this function. Thus we can only guarentee the general case where $H(X) \geq H(Y)$

\section*{2.4 Entropy of functions of a random variable.}
Justify the steps showing that a function $g$ of a random variable $X$ has no
more entropy than the $X$ itself.
\begin{eqnarray}
H(X,g(X))  &=& H(X) + H(g(X) | X) \label{eq: 2.4a} \\
    &=& H(X) \label{eq: 2.4b} \\
    H(X,g(X)) &=& H(g(X))+H(X|g(X)) \label{eq: 2.4c} \\
    &\geq& H(g(X)) \label{eq: 2.4d}.
\end{eqnarray}
\pref{eq: 2.4a} is justified by the chain rule.
\pref{eq: 2.4b} is justified because $g(X)$ is known for any value of $X$,
and hence $H(g(X)|X)=0$ . 
\pref{eq: 2.4c} is just a differently ordered application of the chain rule.
\pref{eq: 2.4d} is justified because entropy (including conditional entropy) is always greater
than zero.

\section*{2.10 Entropy of disjoint mixture}
\begin{equation*}
X=\begin{cases} 
    X_1 & \text{ with probability } \alpha \\
    X_2 & \text{ with probability } (1-\alpha) 
\end{cases} 
\end{equation*}
\textbf{(a)} What is $H(X)$ in term of $H(X_1), H(X_2), \alpha$? 
\begin{equation}
    \theta = \theta(X) = 
        \begin{cases}
            1 & \text{ when } X=X_{1}\\ 
            2 & \text{ when }  X = X_{2} \end{cases}
    \label{eq: theta of X}
\end{equation}
\(
	H(X) = H(X_\theta,\theta) = H(\theta) + H(X|\theta) \\
	= H(\theta)+p(\theta = 1)\ H(X|\theta=1)+p(\theta=2)\ H(X|\theta=2) \\
	= H(\alpha) + \alpha\ H(X_{1}) + (1-\alpha)\ H(X_{2}) \)

where 
\(H(\alpha) = -\alpha\ log\ \alpha - (1 - \alpha )\ log\ (1-\alpha)\)
             
\section*{2.12 Example of joint entropy.}
We are given that  \( p(x,y)=\left[ \frac{1}{3},\frac{1}{3};0,\frac{1}{3} \right] \) Find the following:

\textbf{(a)}Compute  \(H(X)\) and \( H(Y)\):~~
\(
H(X) = H(\frac{2}{3},\frac{1}{3})
= \frac{2}{3} log(\frac{3}{2}) + \frac{1}{3}log(3)\\
\approx 0.9183 bits \) and \(
H(Y) =  H(\frac{2}{3},\frac{1}{3})\
\approx 0.9183 bits
\)
              
\textbf{(b)} Compute \(H(X|Y)\) and \(H(Y|X)\):~~
\(
H(X|Y) = \frac{1}{3}H(X|Y=0) + \frac{2}{3}H(X|Y=1)
=\frac{1}{3}H(\frac{1}{2},\frac {1}{2})+\frac{2}{3}H(0,1)
=\frac{1}{3}1+\frac{2}{3}0 
= \frac{1}{3} bits \) 
                
\(
H(Y|X) = \frac{2}{3}H(Y|X=0)+\frac{1}{3}H(Y|X=1)
=\frac{2}{3}H(\frac{1}{2},\frac{1}{2}) + \frac{1}{3}H(0,1)
=\frac{2}{3} bits
\)
              
\textbf{(c)}
\(
H(X,Y)=H(X)+H(Y|X)
\approx 1.585 bits
\)
          	  
\textbf{(d)} \(H(Y) - H(Y|X) \approx 0.251 bits\)
            	
\textbf{(e)} \( I(X;Y) =H(Y) - H(Y|X) \approx 0.251 bits
\)
		
\textbf{(f)}  Imagine a two circle venn. $H(X,Y)$ represents the overlapped portion plus the two independent portions. $H(X)$ is one circle and $H(Y)$ is the other. The overlapped portion is represented by either $I(X;Y)$ or $H(Y)-H(Y|X)$.
          
\section*{2.15 Data processing} Markov chains: $X_{1}\rightarrow X_{2}\rightarrow \cdots \rightarrow X_{n}$, have joint probabilities of the following form: 
\(
p(x_{1},x_{2},\ldots,x_{n})=p(x_{1})p(x_{2}|x_{1})\cdots p(x_{n}|x_{n-1})
\)
This means that for mutual information we can do some nice reduction.
first let \( X_{3},\ldots,X_{n} = Y\) then you get \(
I(X_{1};X_{2},Y) = I(X_{1};X_{2})+I(X_{1};Y|X_{2})  \)
Now since this is a markov chain, and we know that $X_{1}$ and $Y$ are completely independent
we can say that $I(X_{1};Y|X_{2})=0$ because $X_1$ and $Y$ are conditionally independent given $X_2$. 
Thus the whole equation simplifies to $I(X_{1};X_{2})$

\section*{2.26 Proof of nonnegativity of relative entropy.}
\textbf{(a)} $ln(x) \leq x-1$ for $0\leq x \leq \infty$\\
If the second derivative of a function is positive along its whole domain, 
then that function is convex. 
If it is zero anywhere, then we can't make this claim, 
and the converse isn't necessarily true.
Let \(f(x)=x-1-ln\ x \) then 
\(
f'(x)= 1-\frac{1}{x}\) and \(
f''(x)=\frac{1}{x^2} > 0
\)
so we can say that $f(x)$ is strictly convex and a local minimum is also a global minimum. 
To find the local minimum we just set $f'(x)=0$ and get $x=1$. 
So $f(x) \geq f(1)$ which means that $x-1-ln\ x \geq 1-1-ln\ 1 = 0$. 
Showing that $x-1 \geq ln\ x$ on the interval $(0,\infty)$.
				 
\textbf{(b)} Justify the following steps. 
Let A be the set of $x$ such that $p(x) > 0$
\(
-D_e(p||q)=\sum_{x\in A} p(x) ln \frac{q(x)}{p(x)}
\leq \sum_{x\in A} p(x)\left( \frac{q(x)}{p(x)}-1\right)
=\sum_{x\in A} q(x) - \sum_{x\in A} p(x)
\leq 0
\)
The first step follows from the definition of D, the second step follows from the inequality $ln\ t \leq t - 1$, the third from expanding the sum, and the last step from the fact that the $q(A) \leq 1$ and $p(A)=1$.
				
\textbf{(c)} Conditions for equality?
We have the inequality $ln\ t \leq t-1$ from a previous problem, and showed equality iff $t=1$. Therefore we have equality if $\frac{q(x)}{p(x)}=1$ for all $x\in A$. This implies that $p(x)=q(x)$ for all $x$ and then we'll have equality in the last step as well. 	
          
\section*{2.32 Fano}
\(P(X,Y) = \left( \begin{array}{ccc}
\frac{1}{6} & \frac{1}{12} & \frac{1}{12} \\
\frac{1}{12} & \frac{1}{6} & \frac{1}{12} \\
\frac{1}{12} & \frac{1}{12} & \frac{1}{6} \end{array} \right)\)
 where $X$ is indexed $[a,b,c]$ and $Y$ is indexed $[1,2,3]$. Let $\hat{X}(Y)$  be an estimator for $X$ based on $Y$ and let $P_e=Pr\{\hat{X}(Y)\neq X\}$

\textbf{(a)}
From inspection we see that 
\(\hat{X}(y)=\left\{ \begin{array}{cc}
1 & y=a\\
2 & y=b\\
3 & y=c \end{array}\right. \)
and the associated $P_e$ is the sum of 
\[
P(1,b),  P(1,c), P(2,a),
P(2,c), P(3,a), P(3,b) \] Therefore $P_e=\frac{1}{2}$

\textbf{(b)}
From Fano's inequality we know \(P_e \geq \frac{H(X|Y)-1}{log\ |\chi|}\). \(
H(X|Y)= H(X|Y=a)Pr\{y=a\}+H(X|Y=b)Pr\{y=b\}+H(X|Y=c)Pr\{y=c\}
=H\left(\frac{1}{2},\frac{1}{4},\frac{1}{4}\right)Pr\{y=a\}+H\left(\frac{1}{2},\frac{1}{4},\frac{1}{4}\right)Pr\{y=b\}+H\left(\frac{1}{2},\frac{1}{4},\frac{1}{4}\right)Pr\{y=c\}
=H\left(\frac{1}{2},\frac{1}{4},\frac{1}{4}\right)\left( Pr\{y=a\}+Pr\{y=b\}+Pr\{y=c\}\right)
=H\left(\frac{1}{2},\frac{1}{4},\frac{1}{4}\right)
=1.5 \) bits. 

Our estimator of error $P_e\geq \frac{1.5-1}{log\ 3}=0.316$ is not very close to Fano's bound in this form. Since $\hat{X}\in \chi$ we can use the stronger form of Fano's inequality 
\( P_e \geq \frac{H(X|Y)-1}{log(|\chi|-1)}  \) to get \( P_e \geq \frac{1.5-1}{log\ 2}=\frac{1}{2} \) which is a great bound on our actual $P_e$.

\section*{3.1 Probabilistic inequalities} % HW2
$\mathbf{(a) Markov's Inequality.}$
For non-negative random variable $X$ and  $t>0$,
Show
\begin{equation}
    \Pr\{X \geq t\} \leq \frac{\E X}{t} \label{eq: markovs ineq}
\end{equation}
And provide an example random variable for with the above condition is met
with equality.
\proof
\begin{eqnarray}
    \E X %&= \int_{-\infty}^\infty x f(x)dx &\text{(Defn of expectation.)}
        %\nonumber \\
    &\geq \int_t^\infty x f(x)dx &\text{(Integral over subdomain)} 
    \nonumber \\
    &\geq \int_t^\infty t f(x)dx &\text{($x \geq t$ in integral)} 
        \nonumber \\
    &= t \Pr\{X \geq t\}  &\text{($t$ constant in integral)}
        \label{eq: 3.1a QED}
\end{eqnarray}
The last equation \pref{eq: 3.1a QED} proves Markov's Inequality
\pref{eq: markovs ineq}. $\Box$

Let $X$ be a \emph{random variable} which has value zero with probability
one.
The expected value of $X$ is zero and thus, 
\pref{eq: markovs ineq} is always an equality.

$\mathbf{(b) Chebychev's Inequality.}$
Let $Y$ be a random variable with mean $\mu$ and
variance $\sigma^2$. By letting $X = (Y-\mu)^2$, show that for
$\varepsilon > 0$
\begin{equation}
    \Pr \{(Y-\mu)^2 > \varepsilon^2\} \leq \frac{\sigma^2}{\varepsilon^2}
    \label{eq: chebychevs ineq}
\end{equation}
\proof
\begin{eqnarray}
    \Pr\{(Y-\mu)^2 > \varepsilon^2\} &\leq& \Pr\{(Y-\mu)^2 \geq \varepsilon^2\}
        \nonumber \\
        &=& \Pr\{|Y-\mu| \geq \varepsilon\}
        \label{eq: equate probs} \\
    \Pr \{(Y-\mu)^2 > \varepsilon^2\} &\leq& \frac{\E \big[(Y-\mu)^2\big]}{\varepsilon^2}
        \label{eq:instatiate 3.1a}
\end{eqnarray}
The final inequality coming from an application of Markov's Inequality
\pref{eq: markovs ineq} and the derived inequality \pref{eq: equate probs}.
The numerator $E[(Y-\mu)^2]$ is equal to $\sigma^2$, from the definition
of variance and the fact that $\mu = E[Y]$. Thus we are left with
Chebyshev's inequality \pref{eq: chebychevs ineq}. $\Box$

$\mathbf{(c) The weak law of large numbers.}$
Let $Z_1,Z_2,\cdots,Z_n$ be a sequence of i.i.d. random variables with mean $\mu$ and variance $\sigma^2$. Let
$\overline{Z_n} = \frac{1}{n}\sum{i=1}^n Z_i$ be the sample mean.
Show 
\begin{equation}
    \Pr \{|\overline{Z_n} - \mu| > \varepsilon\} \leq
    \frac{\sigma^2}{n\varepsilon^2}
    \label{eq: weak loln}
\end{equation}
\proof
$\overline{Z_n}$ is a random variable with mean $\mu$ and variance
$\frac{\sigma^2}{n}$. This is easily computed because expectation
distributes across linear combinations, 
%\begin{equation*}
%    E\bigg[\frac{1}{n}\sum_{i=1}^{n}Z_i\bigg] 
%    = \frac{1}{n} \sum_{i=0}^{n} E[Z_i] = \frac{n\mu}{n} = \mu
%\end{equation*}
, and variance distributes across linear combinations, squaring the
leading coefficients,
%\begin{equation*}
%    Var\bigg[\frac{1}{n}\sum_{i=1}^{n}Z_i\bigg]
%    = \frac{1}{n^2}\sum_{i=0}^{n} Var[Z_i] 
%    = \frac{n\sigma^2}{n^2} 
%    = \frac{\sigma^2}{n}
%\end{equation*}
. The weak law of large numbers \pref{eq: weak loln} comes from applying
Chebychev's inequality \pref{eq: chebychevs ineq} to $\overline{Z_n}$.
$\Box$

\section*{3.2 AEP and mutual information.}
Let $(X_i,Y_i)$ be i.i.d. $\sim p(x,y)$. We form the log likelihood ratio
of the hypothesis that $X$ and $Y$ are independent vs. the hypothesis that
$X$ and $Y$ are dependent. What is the limit of
\begin{equation}
    \frac{1}{n}\log \frac{p(X^n)p(Y^n)}p(X^n,Y^n)
    \label{eq: normalized llr}
\end{equation}
$Solution$:
\begin{eqnarray}
    \frac{1}{n}\log \frac{p(X^n)p(Y^n)}{p(X^n,Y^n)}
    &=& \frac{1}{n} \sum_{i=1}^{n} \log \frac{p(X_i)p(Y_i)}{p(X_i,Y_i)}
    \label{eq: log of prods} \\
    &\rightarrow& -I(X;Y) 
    \label{eq: from MI defn}
\end{eqnarray}

The independence assumption makes \pref{eq: normalized llr} the log of a
product, yielding the sum of logs in \pref{eq: log of prods}.
The convergence in \pref{eq: from MI defn} comes from the convergence of
\pref{eq: log of prods} to $\E \log \frac{p(X_i)p(Y_i)}{p(X_i,Y_i)}$ by
the (strong?) law of large numbers. \pref{eq: from MI defn} comes from the
definition of mutual information.

This implies convergence of the likelihood ratio
$\frac{p(X^n)p(Y^n)}{p(X^n,Y^n)} \rightarrow 2^{-nI(X;Y)}$ which is 1 iff
$X$ and $Y$ are independent.
\section*{3.6 AEP-like limit.}
Let $X_1,X_2,\dots$ be i.i.d. drawn according to probability mass function
$p(x)$. Find the limit $\lim_{n \to \infty}
(p(X_1,X_2,\dots,X_n))^{\frac{1}{n}}$.
$Solution$:
\begin{eqnarray}
    P_n &=& (p(X_1,X_2,\dots,X_n))^{\frac{1}{n}}
        \label{eq: defn P_n} \\
    \lim_{n\to \infty} P_n &=& 2^{lim_{n\to\infty}log\ P_n}
        \label{eq: limit of composition} \\
    &=& 2^{-H(X)},
        \label{eq: P_n limit} 
    \end{eqnarray} 
    The convergence of the exponent in \pref{eq: limit of composition} to
    $-H(X)$ comes from the definition of the AEP.
    The above result assumes that $H(X)$ exists.
    \section*{3.11 Proof of bound for $|B_\delta^{(n)}|$}
\textbf{(a)}
Given any two sets $A$, $B$ such that $\Pr{A} > 1-\varepsilon_1$ and
$\Pr{B} > 1-\varepsilon_2$, show that $\Pr(A\cap B) > 1 - \varepsilon_1 -
\varepsilon_2$. Hence
\begin{equation}
    \Pr(A_\varepsilon^{(n)} \cap B_\delta^{(n)}) > 1 - \varepsilon - \delta
    \label{eq: 3.11(a) conclusion}
\end{equation}
\proof
\begin{eqnarray}
    Pr(A\cap B) &=& Pr(A) + Pr(B) - Pr(A\cup B)
    \label{eq: pr A intersect B} \\
    &>& 2 - \varepsilon_1 - \varepsilon_2 - Pr(A\cup B)
    \label{eq: sub hypotheses} \\
    &>& 1 - \varepsilon_1 - \varepsilon_2
    \label{eq: 3.11a}
\end{eqnarray}
\textbf{(b)} 
Inequality \pref{eq: 3.11(a) conclusion} gives $1-\varepsilon-\delta \leq \Pr(A_\varepsilon^{(n)} \cap B_\delta^{(n)})$. 
The definition of the probability of an event makes the RHS equal to
$\sum_{A_\varepsilon^{(n)} \cap B_\delta^{(n)}} p(x^n)$.
This value is at most 
$\sum_{A_\varepsilon^{(n)} \cap B_\delta^{(n)}} 2^{-n(H-\varepsilon)}$, 
by the definition of AEP 
(for $x^n \in A_\varepsilon^{(n)}$ $p(x^n) \leq 2^{-n(H - \varepsilon)}$).
The sum in carried out over a constant summand, and is equal to
$|A_\varepsilon^{(n)} \cap B_\delta^{(n)}|2^{-n(H-\varepsilon)}$, which is
upperbounded by $|B_\delta^{(n)}|2^{-n(H-\varepsilon)}$, because the
cardinality of a set intersection is at most the cardinality of either
set.


\section*{4.1 Doubly stochastic matrices.} % HW3
A \emph{doubly stochastic matrix} $P$ is square with entries in $[0,1]$ where the columns sum to one as well as the rows. 
\textbf{(a)}  Let b = aP, show H(b) $\ge$ H(a). 
\textbf{(b)} Show that the stationary distribution $\mu$ for a doubly stochastic matrix P is the uniform distribution. 
\textbf{(c)} Conversely, prove that if the uniform distribution is a stationary distribution for a Markov transition matrix P , then P is doubly stochastic.
$Solutions$:
\textbf{(a)}
\begin{eqnarray*}
H(b) - H(a) 
= -\sum_j b_j \log (b_j) + \sum_i a_i \log (a_i) \\
= - \sum_j \sum_i a_iP_{ij} \log(b_j)+\sum_i a_i \log (a_i) \\
= \sum_{ij}a_iP_{ij} \log(\frac{ a_i }{ b_j }) 
\text{ (mult by $\sum_j P_{ij} = 1$)} \\
\ge (\sum_{ij}a_iP_{ij}) \log(\frac{\sum_{ij} a_i }{\sum_{ij} b_j })
\text{ (by log sum ineq\pref{thm: log sum})} \\
\end{eqnarray*}

\textbf{(b)} If the matrix is doubly stochastic, the substituting $u_i$ = $1\over{n}$, we can easily see it satisfies $u=uP$
\textbf{(c)} if the uniform is a stationary distribution, then
$\frac{1}{n} = u_i = \sum_{j}u_{j}P_{ij}$ = $ \frac{1}{n}\sum_jP{ji}$ or $\sum_jP{ij}=1$ or that the matrix is doubly stochastic.


\section*{4.6 Monotonicity of entropy per element.}
For a stationary stochastic process $X_1, X_2, \cdots , X_n$, show that
\\ \textbf{(a)} $\frac{H(X_1,X_2,\cdots,X_n)}{n} \le \frac{H(X_1,X_2,\cdots,X_{n - 1})}{ n - 1} $
\\ \textbf{(b)} $\frac{H(X_1,X_2,\cdots,X_n)}{n} \ge H(X_1,X_2,\cdots,X_n)$

$Solutions$:
\begin{eqnarray*}
\frac{H(X_1,X_2,\cdots,X_n)}{n} = \frac{\sum_{i=1}^n H(X_i | X^{i-1})}{n} \\ 
= \frac{(H(X_n | X^{n-1}) + \sum_{i=1}^{n-1} H(X_i |X^{i-1}) } {n} \\ 
= \frac{H(X_n | X^{n-1}) + H(X_1,X_2,\cdots,X_{n-1}) } {n} \\
\end{eqnarray*}
Sationarity implies for $1\le i \le n$, $H(X_n | X^{n-1}) \le H(X_i | X^{i-1}) $
which further implies.
\begin{eqnarray*}
H(X_n | X^{n-1}) &\le& \frac{ \sum_{i=1}^{n-1} H(X_i | X^{i-1})} {n-1} \\
&=& \frac{ H(X_i,X_2,\cdots,X_{n-1})} {n-1}
\end{eqnarray*}
combining past two results yeilds, 
\begin{eqnarray*}
\frac{H(X^n)}{n} \leq \frac{H(X^{n-1})}{n} \bigg[ \frac{ 1 } {n-1}  + 1 \bigg]
= \frac{H(X^{n-1})}{n-1}
\end{eqnarray*}
\\
\textbf{(b)} Stationarity implies $\forall 1 \leq i \leq n$, $H(X_n |
X^{n-1} ) \leq H(X_i | X^{i-1})$. Thus,
\begin{eqnarray*}
H(X_n | X^{n-1}) \leq \frac{\sum_{i=1}^n H(X_n | X^{n-1}) } {n}
= \frac{H(X^n)}{n}.
\end{eqnarray*}


\section*{4.7 Entropy rates of Markov chains.}

Let matrix P = $[ (1-p_{01}, p_{01}), (p_{10}, 1-p_{10}]$ 
\textbf{(a)} The stationary distribution is: $u_0=\frac{p_{10}}{p_{01}+p_{10}}$, $u_1=\frac{p_{01}}{p_{01}+p_{10}}$
Therefore the entropy is: \( H(X_2 \mid X_1) = u_0H(p_{01})+u_1H(p_{10}) = \frac{p_{10}H(p_{01})+ p_{01}H(p_{10})}{p_{01}+p_{10}}. \)

\textbf{(b)} The entropy rate is at most 1 bit because process has only two states. This rate can be achieved iff \(p_{01}=p_{10}=1/2 \), in which case the process is i.i.d with \( \Pr(X_i = 0) =\Pr(X_i=1) = 1/2 \)

Let matrix P = $[ (1-p, p), (1, 0)]$ 
\textbf{(c)} special case if the general two-state Markov chain, the entropy rate is \( H(X_2 | X_1)=u_0H(p)+u_1H(1)=\frac{H(p)}{p+1} = \frac{-p\log(p)-(1-p)\log(1-p)}{1+p} \)
\textbf{(e)} By calculus: \( p=\frac{(3-5^{1/2})}{2}=0.382.\) the max value is \(H(p)=H(1-p)=H(\frac{5^{1/2}-1}{2}) = 0.694bits\)
note that \( \frac{(5^{1/2}-1)}{2} =.618\) is (the reciprocal of) the Golden Ratio.

\section*{4.9 Initial conditions.}
Show for the initial conditions of a markov chain become more difficult to
recover as the future $X_n$ unfolds. That is $H (X_0|X_n) \ge H (X_0|X_{n-1})$.

$Solution$: For a Markov chain, the data processing inequality
\pref{thm: data processing} states $I(X_0  ; X_{n-1}) \ge I(X_0 ; X_n)$. 
Therefore $H(X_0) - H(X_0 | X_{n-1}) \ge H(X_0) - H(X_0 | X_n)$. This
shows that $H(X_0|X_n)$ increases with $n$.

\section*{4.11 Stationary processes.}
Let	$\cdots, X_{-1}, X_0, X_1, \cdots$ be	a stationary (not necessarily
Markov) stochastic process. Either prove the following statements or give
a counter-example.
\begin{eqnarray}
    H(x_n | X_0) = H(X_{-n} | X_0) \label{eq: 4.11a}\\
    H(X_n\mid X_0) \ge H(X_{n-1} \mid X_0) \label{eq: 4.11b}\\
    H(X_n \mid X_1^{n-1}, X_{n+1}) \text{ is non-increasing in n.} \label{eq: 4.11c}
\end{eqnarray}

Solution: \textbf{Equation \pref{eq: 4.11a}} True. $H(X_n | X_0) = H(X_n,X_0) - H(X_0)$ and $H(X_{-n} | X_0) = H(X_{-n},X_0) - H(X_0)$ by stationarity.

\textbf{Equation \pref{eq: 4.11b}} Not true in general. True for first order markov chains. A simple counterexample is a periodic process with period n. Let $X_0,X_1,X_2,\cdots,X_{n-1}$ be i.i.d. uniformly distributed binary random variables and let \(X_k =X_{k-n}\) for \( k \ge n \) In this cases, $H(X_n \mid X_0)$ and $H(X_{n-1} \mid X_0) = 1$, contradicting the statement $H(X_n | X_0) \ge H(X_{n-1} \mid X_0)$.

\textbf{Equation \pref{eq: 4.11c}} True. By stationarity 
\(H(X_n \mid X_1^{n-1}, X_{n+1}) 
= H(X_{n+1} \mid X_2^{n}, X_{n+2}) 
\ge H(X_{n+1} \mid X_1^{n}, X_{n+2}) \). The inequality follows from the fact that conditioning reduces entropy.

\section*{Example problems}

\textit{Entropy Rate of Typewriter:} 
All letters are equally likely. 
$m$ letters, sequence has length $n$, 
$m^n$ sequences with equal probability ($p(x^n) = \frac{1}{m^n}$): 
$H(X^n) = \sum_{x^n} p(x^n) log(\frac{1}{p(x^n)}) =  \frac{m^n}{m^n}log(m^n) = n*log(m)$; $H(\chi) lim_{n \to \infty} \frac{1}{n} H(X^n) = lim_{n \to \infty} log(m) = log(m)$ bits per symbol.

\textit{Entropy Rate of i.i.d. RVs:} 
$X_i$ is i.i.d: 
$H(\chi) = lim_{n \to \infty} \frac{1}{n} H(X^n) = lim_{n \to \infty} \frac{nH(X_1)}{n} = H(X_1)$

\textit{Entropy Rate of Independent RVs:} 
$X_i$s independent implies $H(X^n) = \sum_i^n H(X_i)$; 
$H(\chi) = lim_{n \to \infty} \frac{1}{n} \sum_i^n H(X_i) $; 
Since $H(X_i)$s are different, 
this entropy does not exist (and may not converge), 
i.e. $P(X_i = 1) = r_i, P(X_i = 0) = 1 - r_i$, 
where $r_i$ is not a constant value and is a function of $i$. 
$r_i = 0.5$ if $2k < log(log(i)) \leq 2k + 1$, $r_i = 0$ if $2k + 1 < log(log(i)) \leq 2k +2 $ for some constant $k$. $H(X_i)$ switches between 1 and 0, $H(\chi)$ changes with $n$.

\textit{Shuffling Does Not Decrease Entropy:} 
Let $X =$ location of cards, $T =$ shuffle operation of the cards, 
where $X$ and $T$ are independent. Show $H(TX) \geq H(X)$: $H(TX) \geq H(TX|T)$ (conditioning reduces entropy) $= H(T^{-1}TX|T)$ (if shuffle operation is known, so is inverse) = $H(X|T) = H(X)$ (since $X$ and $T$ are independent).

\textit{Source Entropy per Unit Time} 
Given a discrete memoryless source with alphabet $1 \gets p_1, 2 \gets p_2$ with symbol duration $1 \gets 1$, $2 \gets 2$. 
Maximize the source entropy per nit time: $f(p_1) = \frac{H(p_1)}{E(l(x))}$ (where $l$ is duration), $E(l(x)) = T(p_1) = 1*p_1 + 2*p_2 = 1*p_1 + 2(1-p_1) = 2 - p_1$, $\frac{H(p_1)}{T(p_1)} = \frac{-p_1 log(p_1) - (1-p_1)log(1-p_1)}{2-p_1}$. Note if $p_1 = 1 \to f(p_1) = 0$, $p_1 = 2 \to f(p_1) = 0$ 
(therefore take derivative): $\frac{\partial f(p_1)}{\partial p_1} = \frac{ T(p_1)\frac{\partial H(p_1)}{\partial p_1} - H(p_1)\frac{\partial T(p_1)}{\partial p_1}}{(T(p_1))^2} = \frac{(2-p_1)[-log(p_1) -1 + log(1-p_1) - 1] - H(p_1)(-1)}{(T(p_1))^2}=0$; $ln(1-p_1) - 2ln(p_1) = 0 \to ln(1-p_1) = 2ln(p_1) = ln(p_1^2)$, $1-p_1 = p_1^2 \to p_1^2 + p_1 - 1 = 0 \to p_1 = \frac{1}{2}(\sqrt(5) - 1), p_2 = 1 - p_1 = (\frac{3}{2} - \frac{1}{2}\sqrt(5))$; $\frac{H(p_1)}{T(p_1)} = 0.69424$ bits. 


\end{multicols}
\end{tiny}
\end{document}
