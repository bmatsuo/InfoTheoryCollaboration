\documentclass[10pt]{article}

\usepackage{fullpage,latexsym,amsthm,amsmath,amsfonts,algorithmic,graphicx,subfigure,color,verbatim,epsfig,fancyhdr}
\pagestyle{empty}
% Define new commands.
\DeclareMathOperator*{\E}{\mbox{\text{E}}}

\newcommand{\pref}[1]{{(\ref{#1})}}
\newcommand{\bmat}[1]{{\color{red}{#1}}}
% Define new commands.
\newcommand{\setbrace}[1]{{\{#1\}}}
\newcommand{\setbigbrace}[1]{{\big\{#1\big\}}}
\newcommand{\setbiggbrace}[1]{{\bigg\{#1\bigg\}}}
\newcommand{\annotate}[1]{{\color{red}{#1}}}

\newtheorem{defn}{Definition}[section]
\newtheorem{thm}{Theorem}[section]
\newtheorem{lemma}{Lemma}[section]
\newtheorem{corollary}{Corollary}[section]
\newtheorem{question}{Question}[section]
\newtheorem{proposition}{Proposition}[section] 
\usepackage{qtree}
\usepackage{multicol}
\usepackage{savetrees}
\usepackage{times}
\setlength{\parskip}{0cm}
\setlength{\parindent}{0em}
\usepackage{txfonts}

% Begin the actual document content here.
\begin{document}
\begin{tiny}
\begin{multicols}{3}

\fontsize{4.5pt}{5pt}

\textbf{\scriptsize Entropy, Relative Entropy and Mutual Information}
Definitions, alternate expressions, and properties.
\begin{align}
\underbrace{H(X)}_\text{entropy}&= -\sum_{x\in X} p(x) \log\ p(x) \label{eq:entropy}\\
H(X)& \geq 0\\
H_b(X)&= (\log_b a)H_a(X)\\
H(X|Y)& \leq  H(X)\text{ eq iff ind} \label{eq:conditioning}\\
X^n&=[X_1,\ldots,X_n]\\
H(X^n)& \leq \sum_{i=1}^n H(X_i)\text{ eq iff ind} \label{eq:jointsum}\\
H(X)&\leq  \log |\chi|  \text{ eq iff iid}\label{eq:alphabetentropy}\\
H(p)&\text{ is concave in $p$}\\
\underbrace{D(p||q)}_\text{rel ent} & = \sum_x p(x) \log \frac{p(x)}{q(x)}\\
\underbrace{I(X;Y)}_\text{mut inf} & = \sum_{x\in X} \sum_{y \in Y} p(x,y) \log \frac{p(x,y)}{p(x)p(y)}\\
H(X)&=E_p \log \frac{1}{p(X)}\\
H(X,Y)&=E_p\log \frac{1}{p(x,y)}\\
H(X|Y)&=E_p\log \frac{p(X,Y)}{p(X)p(Y)}\\
D(p||q)&=E_p\log \frac{p(X)}{q(X)}\\
I(X;Y)&=H(X)-H(X|Y)\\
&=H(Y)-H(Y|X)\\
&=H(X)+H(Y)-H(X,Y)\\
I(X;Y,Z) &= I(X;Z) + I(X;Y|Z) \\
&= I(X;Y) + I(X;Z|Y)\\
I(X;Y|Z) &= H(X|Z) - H(X|Y,Z) \\
D(p||q)&\geq 0 \text{ eq iff $p=q$}\\
I(X;Y)&=D(p(x,y)||p(x)p(y))\\
	&\geq 0 \text{ eq iff $p(x,y)=p(x)p(y)$}
\end{align}
Given $|\chi|=m$ and $u$ is the uniform distribution over $\chi$, we get equation~\ref{eq:chimu}.
\begin{align}
D(p||u)&=\log m - H(p)\label{eq:chimu}\\
D(p||q)&\text{ is convex in the pair }(p,q)
\end{align}
Following are chain rules:
\begin{align}
H(X^n)&=\sum_{i=1}^n H(X_i|X^{i-1})\\
I(X^n;Y)&=\sum_{i=1}^n I(X_i;Y|X^{i-1})\\
D(p(x,y)||q(x,y))&=D(p(x)||q(x))\\
&+D(p(y|x)||q(y|x))
\end{align}
{\bf Jensen's Inequality:} if $f$ is a convex function, then
\begin{equation}
Ef(x) \geq f(EX)
\end{equation}
{\bf Log sum inequality:} for n positive numbers $a_1,\ldots,a_n$ and $b_1,\ldots,b_n$,
\begin{equation}
\sum_{i=1}^n a_i \log \frac{a_i}{b_i} \geq \left( \sum_{i=1}^n a_i \right) \log \frac{\sum_{i=1}^n a_i}{\sum_{i=1}^n b_i}
\end{equation}
with equality only if $\frac{a_i}{b_i}$ is constant. 

{\bf Data processing inequality:} given some Markov processes $X\rightarrow Y\rightarrow Z$ we know $I(X;Y)\geq I(X;Z)$

{\bf Sufficient Statistic:} $T(X)$ is sufficient relative to $\{f_\theta (x)\}$ iff $I(\theta;X)=I(\theta;T(X))$ for all distributions on $\theta$.

{\bf Fano's inequality:} Let $P_e=Pr\{\hat{X}(Y)\neq X\}$. Then $H(P_e)+P_e \log |\chi| \geq H(X|Y)$, more general: $P_e \geq \frac{H(X|Y)-1}{log|X|}$.

{\bf Inequality.} If $X$ and $X'$ are independent and identically distributed then $Pr(X=X') \geq 2^{-H(X)}$


\textbf{\scriptsize Channel capacity}
{\bf Channel capacity.} The logarithm of the number of distinguishable inputs is given by $ C= \text{max}_{p(x)} I(X;Y).$

{\bf Common examples} Binary symmetric channel: $C= 1 - H(p)$. Binary erasure channel: $C = 1 - \alpha$. Symmetric channel: $C= \log |\gamma | - H(\text{row of transition matrix})$

{\bf Properties of C}  1. $0 \leq C \leq \text{min}\{\log |\chi| - \log | \gamma | \}$  2. $|(X;Y)$ is a continuous concave function of $p(x)$.

{\bf Joint typicality.}  The set $A_{\epsilon}^{(n)}$ of {\it jointly typical} sequences $\{(x^{n},y^{n})\}$ with respect to the distribution $p(x,y)$ is given by $A_{\epsilon}^{(n)} = \{ (x^{n},y^{n}) \in \sigma^{n} \times \chi^{n}: \left| -\frac{1}{n}\log p(x^{n}) - H(X) \right| < \epsilon, $ $\left| -\frac{1}{n}\log p(y^{n}) - H(Y) \right| <\epsilon, $ $\left| -\frac{1}{n} \log p(x^{n},y^{n})-H(X,Y) \right| <\epsilon \}, $ where $p(x^{n},y^{n}) = \prod_{i=1}^{n}p(x_{i},y_{i}).$


{\bf Joint AEP} Let $(X^{n}, Y^{n})$ be sequences of length {\it n} drawn i.i.d according to $p(x^{n},y^{n})=\prod_{i=1}^{n}p(x_{i},y_{i})$ Then: 1. Pr$((X^{n},Y^{n}) \in A_{\epsilon}^{(n)}\rightarrow 1$ as $n\rightarrow \infty$. 2. $|A_{\epsilon}^{(n)}|\leq 2^{n(H(X,Y)+\epsilon)}$. 3. If $(\tilde{X}^{n}, \tilde{Y}^{n})\sim p(x^{n})p(y^{n}),$ then Pr$\left( (\tilde{X}^{n},\tilde{Y}^{n})\in A_{\epsilon}^{(n)} )\right) $ $\leq 2^{-n(I(X;Y)-2\epsilon)}$

{\bf Channel coding theorem.} All rates below capacity {\it C} are achievable, and all rates above capacity are not; that is, for all rates $R<C$, there exists a sequence of $(2^{nR},n)$ codes with probability of error $\lambda^{(n)}\leftarrow 0$. Conversely for rates $R>C$, $\lambda^{(n)}$ is bounded away from 0.

{\bf Feedback capacity.} Feedback does not increase capacity for discrete memoryless channels (i.e., $C_{\text{F}B} = C$).

{\bf Source-channel theorem.} A stochastic process with entropy rate $H$ cannot be sent reliably over a discrete memoryless channel if $H > C$. Conversely, if the process satisfies the AEP, the source can be transmitted reliably if $H<C$.



\textbf{\scriptsize Other useful laws and properties}
{\bf General Statistical things} Expected value $E(x)=\sum_{i=1}^n x_i p(x_i)$

Variance: $\sigma^2 = E\left[ \left(X-\mu\right)^2\right]$

{\bf Law of Large Numbers} Weak law:
\begin{eqnarray*}
\bar{X}_n\overset{p}\rightarrow \mu \text{ when } n \rightarrow \infty\\
\lim_{n\rightarrow \infty} Pr\left(|\bar{X}_n-\mu| < \epsilon \right) = 1
\end{eqnarray*}
Strong law:
\begin{equation}
Pr\left(\lim_{n\rightarrow \infty} \bar{X}_n = \mu \right) = 1
\end{equation}

{\bf Derivation Rules} Chain rule: $(f \circ g)'(x)=f'(g(x))g'(x)$

Product Rule: $(f\cdot g)' = f' \cdot g + f \cdot g'$

Quotient Rule: \[f(x)=\frac{g(x)}{h(x)}\text{ and }f'(x)=\frac{g'(x)h(x)-g(x)h'(x)}{\left(h(x)\right)^2}\]

%\textbf{\scriptsize Channel Capacity Problems}
%{\bf Channels with memory have higher capacity} {\it problem} Consider a binary symmetric channel with $Y_{i} = X_{i}\oplus Z_{i}$, where $\oplus$ is mod 2 addition , and $X_{i},Y_{I} \in \{0,1\}$. Suppose that $\{Z_{i}\}$ has constant marginal probabilities Pr$\{Z_{i}=1\} = p = 1-\text{Pr}\{Z_{i}=0\}$, but that $Z_{1}, Z_{2}, \ldots Z_{n}$ are not necessarily independent. Assume that $Z^{n}$ is independent of the input $X^{n}$ Let $C=1-H(p,1-p)$ Show that $\text{max}_{p(x_{1},\ldots,x_{n})}I(X_{1},\ldots,X_{n};Y_{1},\ldots,Y_{n}) \geq nC$

{\it solution} 
\begin{eqnarray*}
I(X_{1},\ldots,X_{n};Y_{1},\ldots,Y_{n})\\
= H(X_{1},\ldots,X_{n})-H(X_{1},\ldots,X_{n}|Y_{1},\ldots,Y_{n})\\
= H(X_{1},\ldots,X_{n}) - H(Z_{1},\ldots,Z_{n}|Y_{1},\ldots,Y_{n})\\
\geq H(X_{1},\ldots,X_{n}) - H(Z_{1},\ldots,Z_{n})\\
\geq H(X_{1},\ldots,X_{n}) - \sum H(Z_{i})\\
=H(X_{1},\ldots,X_{n}) - nH(p)\\
=n - nH(p)
\end{eqnarray*}
if $X_{1},\ldots,X_{n}$ are chosen i.i.d $\sim$ Bern$\left( \frac{1}{2}\right)$. The capacity of the channel with memory over n uses of the channel is $n$


\textbf{\scriptsize Proof of Shannon Code shortness}
{\it Let $l(x)$ be the codeword lengths associated with the Shannon code, and let $l'(x)$ be the codeword lengths associated with any other uniquely decodable code. Then} Pr$(l(X) \geq l'(X) + c)\leq \frac{1}{2^{c-1}}$

For example, the probability that $l'(X)$ is 5 or more bits shorter than $l(X)$ is less than $\frac{1}{16}$. 

{\bf Proof}
\begin{eqnarray*}
\text{Pr}(l(X) \geq l'(X) + c)\\
=\text{Pr}\left(\lceil \log \frac{1}{p(X)} \geq l'(X) + c \right)\\
\geq \text{Pr}\left( \log \frac{1}{p(X)} \geq l'(C) + c - 1 \right)\\
=\text{Pr}\left( p(X) \leq 2^{-l'(X)-c+1}\right)\\
=\sum_{x:p(x)\leq 2^{-l'(x)-c+1}} p(x)\\
\leq \sum_{x:p(x)\leq 2^{-l'(x)-c+1}}2^{-l'(x)-(c-1)} \\
\leq \sum_{x}2^{-l'(x)}2^{-(c-1)}\\
\leq 2^{-(c-1)}
\end{eqnarray*}
Since $\sum 2^{-l'(x)} \leq 1$ by the Kraft inequality.

\begin{align*}
\chi &= \text{State space } (X_i \in \chi)\\
X^n &=(X_1,\ldots,X_n)
\end{align*}


%%%%%%%%%%%Begin Data Compression%%%%%%%%
\textbf{\scriptsize Data Compression}
Definitions, alternate expressions, and properties.
\begin{align}
    \underbrace{L(C)}_\text{Exp. Length}&= \sum_{x\in X} p(x)l(x) \label{defn: Exp. Length}\\
C(X_1,X_2,\ldots,X_n) &= C(X_1)C(X_2)\ldots C(X_n)
\end{align}

\begin{thm}{Kraft's Inequality}\label{thm: kraft}
For any instantaneous code (prefix code), the codeword length $l_1,l_2 \ldots , l_m$ must satisfy
\end{thm}

\begin{equation}
\sum_i D^{-li}  \leq 1.
\end{equation}

\section*{Optimal Codes}
Minimize $L = \sum p(x) l(x)$ subject to constraint $D^{-l_i} \leq 1$. So minimize the following equation w.r.t. $l_i$,
\begin{align*}
\underbrace{J}_\text{Lagrange min.} &=& \sum {p_i}{l_i} + \gamma \sum D^{-l_i} \label{defn: lagrange} \\
\bigg(\frac{\partial J}{\partial l_i} = 0\bigg) &\Rightarrow& \bigg(D^{-l_i} = \frac{p_i}{\gamma \log_e D}\bigg) \\
&\Rightarrow& \gamma \ge \frac{1}{\log_e D} \text{ (plug into Kraft ineq.)} \\
&\Rightarrow& l_i^* \geq - \log_D p_i \text{ \bigg(eq when $\gamma$ equals $\frac{1}{\log_e D}$\bigg)}
\end{align*}

\begin{thm}{Expected Length and Entropy}\label{thm: codewordentropy}
The expected length L is greater than or equal to the entropy (instant code).
\end{thm}

\begin{eqnarray}
L \geq H_D(X) \text{ (equal iff } D^{-l_i} = p_i \text{)} \\
L - H_D(X) = \sum {p_i}{l_i} - \sum {p_i} \log_D \frac {1}{p_i}\\
= - \sum {p_i} \log_D D^{-l_i} + \sum {p_i} \log_D {p_i} \\
\text{letting } {r_i} = \frac {D^{-l_i}}{\sum_j D^{-l_j}} \text{ and } c = \sum D^{-l_i} \\
L - H = \sum {p_i} \log_D \frac {p_i}{r_i} - \log_D C \\
= D(p||r) + \log_D \frac {1}{c} \geq 0 
\end{eqnarray}

%%%%%%%%%%%End Data Compression%%%%%%%%


%%%%%%%%%%%Begin Optimal Code Lengths%%%%%%%%

\textbf{\scriptsize Bounds on Optimal Code Length}
\begin{eqnarray*}
H(X) \leq L < H(X) + 1
\end{eqnarray*}
After finding the D-adic variable r from before. \\*
$L-H_D = D(p||r) - \log $($\sum D^{-l_i}$) \\
${l_i} = \lceil{\log_D \frac {1}{p_i}} \rceil$ \\
$\sum D^{- \lceil \log \frac {1}{p_i}\rceil}\leq\sum D^{- \log\frac{1}{p_i}}=\sum {p_i}=1$ 
\begin{thm}{Optimal Codeword Lengths} \label{thm:optimum codewords}
Let ${l_1^{*}}$, ${l_2^{*}}$ \ldots , ${l_m^{*}}$be optimal codeword lengths for a
source disribution p and a D-ary alphabet, and let ${L^{*}}$ be the associated
expected length of an optimal code (${L^{*}}=\sum{p_i}{l_i^{*}}$ ). Then
${H_D(X)}\leq L^{*}<{H_D(X)}+1$ 
\end{thm}
Define ${L_n}$ to be the expected codeword length per input symbol, that is,
if ${l(x_1,x_2, \ldots ,x_n)}$ is the length of the binary codeword associated with
${(x_1,x_2, \ldots ,x_n)}$, then \newline
\begin{eqnarray*}
{L_n} = \frac {1}{n}\sum{p(x_1,x_2, \ldots ,x_n)}{l(x_1,x_2, \ldots ,x_n})\\
= \frac{1}{n}El(x_1,x_2, \ldots ,x_n)\\
\text{Now apply the bounds}\\
{H(x^{n})} \leq El(x_1,x_2, \ldots ,x_n) < {H(x^{n})} + 1 \\
\text{Since } {x^{n}} \text{is i.i.d,} {H(X)} \leq {L_n} < {H(X)} + \frac{1}{n} \\*
\end{eqnarray*}
We can now get arbitrarily close to the entropy by increasing n, \newline
${H(x^{n})} \leq El(x_1,x_2, \ldots ,x_n) < {H(x^{n})} + 1$ \\*
$\frac{H(x^{n})}{n}\leq {L_n}<\frac{H(x^{n})}{n}+\frac{1}{n}$ \\*
If the stochastic process is stationary,$\frac{H(x^{n})}{n}$ tends to the entropy rate \newline
and the expected length tends to the entropy as n goes to infinity \\
Note: Shannon Code Assignment $l(x) = \lceil \log \frac {1}{q(x)} \rceil$
\begin{thm}{Wrong Code} \label{thm:wrongcode}
The expected length under ${p(x)}$ of the code assignment ${l(x)}=\lceil\log\frac{1}{q(x)}\rceil$ satisfies 
$H(p)+D(p||q) \leq E_pl(X)<H(p)+D(p||q)+1$
\end{thm}
\begin{proof}
    \begin{eqnarray}
El(x) = \sum_x p(x) \lceil\log \frac{1}{q(x)} \\
< \sum_x p(x) (\log \frac{1}{q(x)} + 1) \\
= \sum_x p(x) \log \frac{p(x)}{q(x)} \frac{1}{p(x)} + 1 \\
= \sum_x p(x) \log \frac{p(x)}{q(x)} + \sum_x p(x) \log \frac{1}{p(x)} + 1 \\
= D(p||q) + H(p) + 1
\end{eqnarray}
\end{proof}

%%%%%%%%%%%End Optimal Code Lengths%%%%%%%%


%%%%%%%%%%%Kraft Inequality for Uniquely Decodable Codes%%%%%%%%

\textbf{\scriptsize Kraft Inequality for Uniquely Decodable Codes}
\begin{thm}{McMillan} \label{thm: mcmillan}
The codeword lengths of any uniquely decodable D-ary code must satisfy the Kraft Inequality (seen many times above) 
\end{thm}
\begin{proof}
Consider ${C^{k}}$, the kth extension of the code. By definition of unique decodability, the kth extension of the code is nonsingular; 
the number of code sequences of length ${n}$ in the kth extension of the code must be no greater than ${D^{n}}$. 
\begin{eqnarray}
l({x_1},{x_2}, \ldots , {x_k}) = \sum_{i=1}^{k} l({x_i})\\
(\sum_{x\in \chi} D^{-li})^{k} = \\*
\sum_{x_1\in\chi}\sum_{x_2\in\chi} \ldots \sum_{x_k\in\chi}  D^{-l({x_1})} \ldots D^{-l({x_k})}\\
=\sum_{{x_1},{x_2}, \ldots , {x_k} \in \chi^{k}}  D^{-l({x_1})}\ldots D^{-l({x_k})} \\
=\sum_{x_k} D^{-l({x^{k}})}
\end{eqnarray}
${l_{max}}$ is the maximum codeword length and a(m) is the number of source sequences
${x^{k}}$ mapping into codewords of length m. The mapping is one-to-one, so $a(m) \leq D^{m}$:
\begin{eqnarray}
(\sum_{x\in\chi} D^{-l(x)})^{k} = \sum_{m=1}^{kl_{max}} a(m)D^{-m}\\
\leq \sum_{m=1}^{kl_{max}} D^{m}D^{-m}\\
=kl_{max}\\
\text{and hence } \sum_j D^{-l_j} \leq (kl_{max})^{\frac{1}{k}}
\end{eqnarray}
Since this inequality is true for all ${k}$, it is true in the limit as ${k} \rightarrow \infty$.
Since ${(kl_{max})^{\frac{1}{k}}} \rightarrow 1$, we have 
$\sum_j D^{-l_j} \leq 1$, which is the Kraft inequality.
\end{proof}

\begin{corollary}A uniquely decodable code for an infinite source alphabet $\chi$ also satisfies Kraft. 
\end{corollary}

$\sum_{i=1}^{\infty}D^{-li} = \lim_{N\rightarrow\infty}\sum_{i=1}^{N}D^{-li} \leq 1$.

%%%%%%%%%%%End Kraft Inequality for Uniquely Decodable Codes%%%%%%%


%%%%%%%%%%%%%%%%Huffman Codes%%%%%%%%%%%%%%%%%
\textbf{\scriptsize Huffman Coding and Optimality}
Just remember 20 questions and expand the tree down one side. "Is X equal to 3?", etc...
\begin{lemma}{Huffman Properties}\label{huffman}
\begin{enumerate}
	\item {\it The lengths are ordered inversely with the probabilities. High p(x) means small l(x).}
	\item {\it The two longest codewords have the same length.}
	\item {\it Two of the longest codewords differ only in the last bit and correspond to the two least likely symbols.}
\end{enumerate}
\end{lemma}
\begin{proof} The proof amounts to swapping, trimming and rearranging...Consider the optimal code ${C_m}$
If ${p_j} > {p_k} \text{,then } {l_j} \leq {l_k}$. Here we swap codewords. Consider $C_m^{'}$ an optimal code ${C_m}$:
Then, 
\begin{eqnarray}
L(C_m^{'}) - L(C_m) = \sum {p_i}{l_i^{'}} - \sum {p_i}{l_i} \\
={p_j}{l_k}+{p_k}{l_j} - {p_j}{l_j} - {p_k}{l_k}\\
=({p_j}-{p_k})({l_k}-{l_j})
\end{eqnarray}
But ${p_j} - {p_k} > 0$, andn since ${C_m}$ is optimal, $L(C_m^{'}) - L(C_m) \geq 0$.
Hence, we must have ${l_k} \geq {l_j}$. Thus, ${C_m}$ itself satisfies property 1(above).
\end{proof}
\textbf{\scriptsize Huffman reductions}
Calculation of the average length $\sum_i {p_i^{'}}{l_i^{'}}$ shows that \\*
$L(p) = L^{*}(p^{'})+p_{m-1} + p_m\\$
We can construct ${p^{'}}$ by merging the codewords for the two lowest probability symbols
${m-1} \text{and} {m}$.  The new code for ${p^{'}}$ has average length
\begin{eqnarray}
L(p^{'}) \\
= \sum_{i=1}^{m-2} p_i l_i + p_{m-1}(l_{m-1} -1 )+p_{m}(l_{m} -1)\\
=\sum_{i=1}^{m} {p_i}{l_i} - {p_{m-1}} - {p_{m}}\\
=L^{*}(p) - {p_{m-1}} - {p_{m}}\\
\text{from this we obtain} \\*
L(p^{'}) + L(p) = L^{*}(p^{'}) + L^{*}(p)\\
(L(p^{'}) -  L^{*}(p^{'})) + (L(p) - L^{*}(p)) = 0\\
L(p^{'}) - L^{*}(p^{'}) \geq 0 \\
L(p) = L^{*}(p)
\end{eqnarray}
\begin{thm}{Optimal Huff} \label{thm: opthuff}
Huffamn coding is optimal; that is, if $C^{*}$ is a huffman code and $C^{'}$ is any other uniquely decodable code,
$L(C^{*}) \leq L(C^{'})$
\end{thm}
%%%%%%%%%%%%%%%End Huffman Codes %%%%%%%%%%%%%%%%


%%%%%%%%%%%%%%%Begin Shannon-Fano-Elias Coding%%%%%%%%%%
\textbf{\scriptsize Shannon-Fano-Elias Coding}
The CDF for ${F(x)}$ is defined as \\* $F(x) = \sum_{a\leq x}p(a)$\\
Consider the modified CDF: \\* $\bar{ F(x)} = \sum_{a < x}p(a) + \frac{1}{2} p(x)$\\
Using the first $l(x)$ bits of $\bar{F}(x)$ as a code for $x$. By definition of rounding off, we have \\*
\begin{eqnarray}
\bar{F}(x) - {\lfloor \bar{F}(x) \rfloor}_{l(x)} < \frac{1}{2^{l(x)}}\\
\text{If } l(x) = \lceil \log \frac{1}{p(x)} \rceil + 1 \text{ then } \\
\frac {1}{2^{l(x)}} < \frac {p(x)}{2} =  \bar{F}(x) - F(x-1)\\
L= \sum_x p(x) (\lceil \log \frac {1}{p(x)} \rceil + 1) < H(X) + 2
\end{eqnarray}

\begin{thm}{Dyadic Stuff} \label{thm: dyadic}
For a dyadic probability mass function $p(x)$, let $l(x) = \log \frac{1}{p(x)}$ be the word lengths of the binary Shannon code for the source, and let $l^{'}(x)$ be the lengths of any other uniquely decodable code for the source. Then \\*
$Pr(l(X) < l^{'}(X)) \geq Pr(l(X) > l^{'}(X))$ \\*
, with equality iff $l^{'}(x) = l(x)$ for all $x$. The code length is thus uniquely competitively optimal.\\*
\end{thm}
\begin{proof}
\begin{eqnarray}
Pr(l^{'}(X) < l(X)) - Pr(l^{'}(X) > l(X)) = \\
\sum_{x:l^{'}(x)<l(x)} p(x) - \sum_{x:l^{'}(x)>l(x)} p(x)\\
=\sum_x p(x)sgn(l(x) - l^{'}(x))\\
=E sgn(l(X) - l^{'}(X)) \\
\leq \sum_x 2^{-l(x)} (2^{l(x)-l^{'}(x)} - 1)\\
=\sum_x 2^{-l^{'}(x)} - \sum_x 2^{-l(x)} \\
=\sum_x 2^{l^{'}(x)} - 1 \leq 1-1 = 0 
\end{eqnarray}
\end{proof}
\begin{corollary}For nondyadic probability mass functions,
$E sgn(l(X) - l^{'}(X) - 1) \leq 0$, where $l(x) = \lceil \log \frac {1}{p(x)} \rceil$ and $l^{'}(x)$ is any other code for the source.
\end{corollary}
%%%%%%%%%%%%%%%End Shannon-Fano-Elias Coding%%%%%%%%%%

\textbf{\scriptsize Shannon coding example}
$l(x)= \lceil \log(\frac{1}{p(x)}) \rceil + 1$
\[
\begin{pmatrix}
      X & P(X) & F(X)  & F(X) & \bar{F}(X)_{2} & l(x) & code  \\
      1 & 0.25 & 0.25 & 0.125 & 0.001 & 3 & 001\\ 
      2 & 0.25 & 0.5 & 0.375 & 0.011 & 3 & 011\\ 
      3 & 0.2 & 0.7 & 0.6 & 0.\overline{10011} & 4 & 1001\\ 
      4 & 0.15 & 0.85 & 0.775 & 0.110 \overline{0011} & 4 & 1100\\ 
      5 & 0.15 & 1 & 0.925 & 0.111\overline{0110} & 4 & 1110
\end{pmatrix}
\]


%%%% SAM's Additions%%%%%%
\textbf{\scriptsize Chapter 5 Problems}
\textit{Huffman Dummy Variables}
Use $1+k(D-1)$ where $D$ is the encoded base, $k$ is the smallest constant such that the expression is $\geq$ the number of variables you are encoding (say $V$). Add $1+k(D-1) - V$ dummy variables.

\textbf{5.1: Uniquely Decodable}
$L = \sum_{i=1}^m p_i l_i^{100}, L_1 = min\{L\}$ over instantaneous codes, $L_2 = min\{L\} $over uniquely decodable codes. We observe that since uniquely decodable codes are a superset of instantaneous codes, we can immediately conclude that $L_2 \leq L_1$, however the inequality can be stricter. Specifically, the Kraft and McMillan inequalities do not depend on the standard expected length definition ($\sum_i p(x_i)l_i$), and apply in the $l_i^{100}$ case as well. Thus, any codeword that minimizes $L_2$ must satisfy McMillan, and thus Kraft. Yet, Kraft also asserts that given a set of lengths that satisfy Kraft, there exists a code. Thus, there exists a code with the same code lengths in $L_1$, and $\fbox{$L_1 = L_2$}$.

\textbf{5.1: Martian Fingers}
$(l_1,l_2,...,l_6) = (1,1,2,3,2,3)$ We note that any uniquely decodable D-ary code must satisfy the Kraft inequality: $\sum D^{-l_i} \leq 1$, or, $D^{-1} + D^{-2} + D^{-3} \leq \frac{1}{2}$. Additionally, we assume that the Martian has an integer number of fingers. We know that for $D=2$ the inequality does not hold, however for $D=3$: $\frac{1}{3} + \frac{1}{9} + \frac{1}{27} = \frac{13}{27} \leq  \frac{13}{26} = \frac{1}{2}$. Thus, a Martian has at least $3$ fingers.

\textbf{5.4: Huffman Coding}
\[
 X =
 \begin{pmatrix}
	x_1 & x_2 & x_3 & x_4 & x_5 & x_6 & x_7 \\
	0.49 & 0.26 & 0.12 & 0.04 & 0.04 & 0.03 & 0.02 \\
 \end{pmatrix}
\]
%\Tree [.$1.0$ [.$0.51$ [.$0.25$ [.$0.13$ [.$0.08$ $x_4$ $x_5$ ] [.$0.05$ $x_6$ $x_7$ ] ] $x_3$ ] $x_2$ ] $x_1$ ]
($1.0$ ($0.51$ ($0.25$ ($0.13$ ($0.08$ $x_4$ $x_5$ ) ($0.05$ $x_6$ $x_7$ ) ) $x_3$ ) $x_2$ ) $x_1$ )
\begin{itemize}
	\item[a.] 
\[
 \begin{pmatrix}
	x_1 & x_2 & x_3 & x_4   & x_5   & x_6   & x_7 \\
	1   & 01  & 001 & 00001 & 00000 & 00010 & 00011 \\
 \end{pmatrix}
\]
	\item[b.] $L = \sum_{i=1}^7 l_i p(x_i) = 0.04*5 + 0.04*5 + 0.03*5 + 0.02*5 + 0.12*3 + 0.26*2 + 0.49*1 = 2.02\textrm{ bits}$

	\item[c.] 
\
%\Tree [.$1.0$  [.$0.25$ [.$0.09$ $x_5$ $x_6$ $x_7$ ] $x_3$ $x_4$ ] $x_2$ $x_1$ ]
($1.0$  ($0.25$ ($0.09$ $x_5$ $x_6$ $x_7$ ) $x_3$ $x_4$ ) $x_2$ $x_1$ )

\[
 \begin{pmatrix}
	x_1 & x_2 & x_3 & x_4 & x_5 & x_6 & x_7 \\
	0   & 1   & 21  & 20  & 222 & 221 & 220 \\
 \end{pmatrix}
\]
\end{itemize}

\textbf{5.6: Bad Codes}

\begin{itemize}
	\item[a.] $ \{0, 10, 11 \} $ This code is a $\fbox{valid}$ Huffman code, which could be generated by the following probabilities $\{ \frac{1}{2}, \frac{1}{4}, \frac{1}{4} \}$, where $D=2$.
	\item[b.] $ \{00, 01, 10, 110 \} $ This code is an $\fbox{invalid}$ Huffman code, as a Huffman code has optimal length, and $110$ can be replaced by $11$ to decrease the expected length.  
	\item[c.] $ \{01, 10 \} $ This code is an $\fbox{invalid}$ Huffman code, as a Huffman code has optimal length and $01$ can be replaced with $0$, while $10$ can be replaced by $1$ to reduce the expected length by $2$.
\end{itemize}

\textbf{5.12: Shannon and Huffman}
\begin{itemize}
	\item[a.]
($1.0$ ($\frac{2}{3}$ ($\frac{1}{3}$ $\frac{1}{4}$ $\frac{1}{12}$ ) $\frac{1}{3}$ ) $\frac{1}{3}$ )
\[
 \begin{pmatrix}
	\frac{1}{3} & \frac{1}{3} & \frac{1}{12} & \frac{1}{4}\\
	0           &  10         & 110          & 111\\
 \end{pmatrix}
\]
	\item[b.] We note that the tree above exhibits a $(1,2,3,3)$ code. However, another tree could have been constructed: 

(.$1.0$ (.$\frac{1}{3}$ $\frac{1}{4}$ $\frac{1}{12}$ ) (.$\frac{2}{3}$ $\frac{1}{3}$ $\frac{1}{3}$ ) ) 

Which has Huffman coding:
\[
 \begin{pmatrix}
	\frac{1}{3} & \frac{1}{3} & \frac{1}{12} & \frac{1}{4}\\
	00           &  01         & 10          & 11\\
 \end{pmatrix}
\] with lengths $(2, 2, 2, 2)$.
	\item[c.] We note that in the first code, the symbol with probability $\frac{1}{4}$ is encoded with $3$ bits, whereas the Shannon code length is $\lceil log(\frac{1}{p(x)}) \rceil = 2$ bits. Thus, there are optimal codes with codeword lengths for some symbols that exceed the Shannon code length.
\end{itemize}

\textbf{\scriptsize 7.3 Channels with mevory have higher capacity.}
    $X_i, Y_i \in \{0,1\}$.
    $Y_i = X_i \oplus Z_i$.
    $Z_i = 1$ w/ prob $p$, 0 otherwise.
    $Z_i$ are not assumed independent of one another, but $Z^n$ is independent of $X^n$.

\begin{eqnarray}
    I(X^n;Y^n) &=& H(X^n) - H(X^n|Y^n) \nonumber \\
    &=& H(X^n) - H(Z^n|Y^n) \nonumber\\
    &\ge& H(X^n) - H(Z^n) \nonumber\\
    &\ge& H(X^n) - \sum H(Z_i) \nonumber\\
    &=& H(X^n) - nH(p) \nonumber\\
    &=& n - nH(p) \label{eq:7.3-MI}
\end{eqnarray}
The last equality happens when if $X_i \sim_{i.i.d} Bernouli\big(\frac{1}{2}\big)$.
The capacity of the channel with memory of $n$ uses of the channel is
\begin{eqnarray*}
    nC^{(n)} &=& \max_{p(x^n)} I(X^n; Y^n) \\
    &\ge& I(X^n;Y^n)_{p(x^n)=Bern(\frac{1}{2})} \\
    &\ge& n ( 1 - H(p)) \text{ (from (\ref{eq:7.3-MI}))} \\
    &\ge& nC
\end{eqnarray*}
So channels which have memory have capacity at least as high a capacity as the channel without memory.

\textbf{\scriptsize 7.4 Channel capacity.}

Consider the channel $Y=X+Z\ (\bmod 11)$, where $X \in \setbrace{0,\dots,10}$ and $Z$, independent of $X$, is distributed as

\begin{equation*}
    Z = {{1,\ 2,\ 3}\choose{\frac{1}{3},\ \frac{1}{3},\ \frac{1}{3}}}
\end{equation*}


\textbf{(a)} Find the capacity.

Consider the mutual information $I(X;Y) = H(Y) - H(Y|X)$. The conditional
entropy $H(Y|X)$ is equal to $H(Z)=\log 3$.
So we have
\begin{eqnarray*}
    I(X;Y) &=& H(Y) - H(Y|X)  \\
    &=& H(Y) - \log 3 
    \leq \log {\vert \mathcal{Y} \vert} - \log 3 \label{eq:7.4-cap}
\end{eqnarray*}
Equality can be achieved, and thus a maximum found, if $Y$ is distributed
uniformily on $\setbrace{0,\dots,10}$. This is achieved when $X$ is
distributed uniformily. That is


\begin{align*}
    &\Pr \setbrace{Y=y} \\
    &= \sum_{z=1}^{3}p(z) \Pr \setbrace{X=x : x+z (\bmod 11) = y} \\
    &= \frac{1}{3}\sum_{z=1}^{3} \Pr \setbrace{X = x : x+z\ (\bmod 11) = y} = \frac{1}{11}
\end{align*}
It is easily verified that when $X$ is uniformily distributed over
$\setbrace{0,\dots,10}$, then the above equality is satisfied.

So it has been shown that expression \ref{eq:7.4-cap} is the capacity of
the channel $Y$.

\textbf{(b)} Find the maximizing $p^*(x)$.

The uniform distibution
\begin{equation*}
    p(x)= \begin{cases}
        \frac{1}{11} & \text{for $x \in \setbrace{0,\dots,10}$} \\
        0 & \text{otherwise}
    \end{cases}
\end{equation*}
See the end of part (a) for details.

\textbf{\scriptsize 7.7 Cascade of binary symmetric channels.}

Consider the process defined by $n$ identical independent binary symmetric
channels,

\begin{eqnarray*}
    X_0 \rightarrow BSC &\rightarrow& \dots
    \rightarrow X_{n-1} \rightarrow BSC \rightarrow X_n. \\
 A &=&
 \begin{pmatrix}
  1-p & p   \\
  p   & 1-p \\
 \end{pmatrix}
\end{eqnarray*}
show the error probability of a cascade of $n$ independent binary symmetric channels. We note that we can diagonalize the matrix and then raise this matrix to the power of $n$ to find a simple expression for the error probability. Using $A$s eigenvectors as columns, we construct $P$ and $P^{-1}$:
\begin{align*}
P &= \begin{pmatrix} 1 & -1 \\ 1 & 1 \end{pmatrix}, P^{-1} = \frac{1}{2}\begin{pmatrix} 1 & 1 \\ -1 & 1 \end{pmatrix} \\
D &= P^{-1}AP \\
    &= \frac{1}{2}\begin{pmatrix} 1 & 1 \\ -1 & 1 \end{pmatrix}  \begin{pmatrix}
        1-p & p   \\
        p   & 1-p \\
    \end{pmatrix}\begin{pmatrix} 1 & -1 \\ 1 & 1 \end{pmatrix} \\
    &= \begin{pmatrix} 1 & 0 \\ 0 & 1 - 2p \end{pmatrix}  \\
A^n &= PD^nP^{-1} \\
    &= \begin{pmatrix} 1 & -1 \\ 1 & 1 \end{pmatrix} \begin{pmatrix} 1 & 0 \\ 0 & (1 - 2p)^n \end{pmatrix} \frac{1}{2}\begin{pmatrix} 1 & 1 \\ -1 & 1 \end{pmatrix} \\
&= \begin{pmatrix}
        \frac{1}{2}(1 + (1-2p)^n) & \frac{1}{2}(1 - (1-2p)^n) \\
        \frac{1}{2}(1 - (1-2p)^n) & \frac{1}{2}(1 + (1-2p)^n) \end{pmatrix}  \\
    &= \begin{pmatrix} 1 - p_n & p_n \\
        p_n & 1 - p_n \end{pmatrix} 
\end{align*} , where  $p_n = \frac{1}{2}(1 - (1-2p)^n)$.

\textbf{\scriptsize 7.11 Time-varying channels.}
Consider a time-varying discrete memoryless channel.

Let $Y_1,\dots,Y_n$ be conditionally independent given $X_1,\dots,X_n$ with distribution $p(\bf{x}, \bf{y}) = \prod_{i=1}^{n}p_i(y_i|x_i)$. Let $\bf{X} = (X_1,\dots,X_n)$ and $Y=(Y_1,\dots,Y_n)$. Find $\max_{p(\bf{x})} I(\bf{X};\bf{Y})$.

Using the same chain of inequalities as in the proof of the converse to the chanel coding theorem. That is,
\begin{eqnarray*}
    I(X^n;Y^n) &=& H(Y^n) - H(Y^n|X^n) \\ 
        &=& H(Y^n) - H(Y_i | Y^{i-1}, X^n)  \\
        &=& H(Y^n) - \sum_{i=1}{n} H(Y_i | X_i)
\end{eqnarray*}

Since by the definition of the channel, $Y_i$ depends only on $X_i$ and is conditionally independent of everything else. Continuing the series of inequalities, we have

\begin{eqnarray*}
    I(X^n;Y^n) &=& H(Y^n) - \sum_{i=1}{n} H(Y_i | X_i) \\
    &\le& \sum_{i=1}^{n} H(Y_i) - \sum_{i=1}^{n}H(Y_i | X_i) \\ 
    &\le& \sum_{i=1}^{n} (1-H(p)),
\end{eqnarray*}

Equality holds if $X_i,\dots,X_n$ are $\sim_{i.i.d.} Bern(\frac{1}{2})$. Hence
\begin{equation*}
    \max_{p(x)} I(X^n; Y^n) = \sum_{i=1}^{n}(1-H(p)).
\end{equation*}

\textbf{\scriptsize 7.13 Erasures and errors in a binary channel.}
Consider a channel with binary inputs that has both erasures and errors. Let the probability of error be $\varepsilon$ and the probability of erasure be $\alpha$ (there is a picture).

\textbf{(a)} Find the capacity of the channel.

Like example in the text, set the input distn. for the two inputs to be $\pi$ and $1-\pi$. Then
\begin{eqnarray*}
    C &=& \max_{p(x)}I(X;Y) \\
    &=& \max_{p(x)}[H(Y) - H(Y|X)] \\
    &=& \max_{p(x)}H(Y) - H(1-\varepsilon-\alpha,\alpha,\varepsilon).
\end{eqnarray*}
$H(Y)$ can not be made equal to $\log 3$, because of erasures. 
\begin{eqnarray*}
    H(Y) &=& H( \pi(1-\varepsilon-\alpha) + (1-\pi)\varepsilon,\alpha, 1 - \dots)\\ %(1-\pi)(1-\varepsilon-\alpha) + \pi\varepsilon ) \\
    &=& H(\alpha) + (1-\alpha)H\bigg(\frac{\pi+\varepsilon -\pi\alpha-2\pi\varepsilon}{1-\alpha}\bigg) \\
    &\le& H(\alpha) + (1-\alpha)
\end{eqnarray*}
with equality iff $\frac{\pi+\varepsilon -\pi\alpha-2\pi\varepsilon}{1-\alpha} = \frac{1}{2}$, happening if $\pi = \frac{1}{2}$.

So the capacity of the channel is
\begin{eqnarray*}
    C &=& H(\alpha) + 1 - \alpha - H(1-\alpha-\varepsilon,\alpha,\varepsilon) \\
    &=& H(\alpha) + 1 - \alpha - H(\alpha) - (1-\alpha) H\bigg(\frac{\varepsilon}{1-\alpha}\bigg) \\
    &=& (1-\alpha) \bigg(1 - H\bigg(\frac{\varepsilon}{1-\alpha}\bigg)\bigg)
\end{eqnarray*}

\textbf{(b)} When $\alpha = 0$ (binary symmetric channel), $C = 1- H(\varepsilon)$

Setting $\alpha = 0$ means $C=1-H(\varepsilon)$.

\textbf{(c)} When $\varepsilon = 0$ (binary erasure channel), $C=1-\alpha$.

\textbf{\scriptsize 7.20 Channel with two independent looks at $Y$.}

Let $Y_1$ and $Y_2$ be conditionally independent and conditionally identically distributed given $X$.

\textbf{(a)} Show that $I(X;Y_1, Y_2) = 2I(X;Y_1) - I(Y_1,Y_2)$
\begin{eqnarray*}
    && I(X;Y_1,Y_2) \\
    &=& H(Y_1,Y_2) - H(Y_1,Y_2 | X) \\
    &=& \sum_{i=1,2} H(Y_i) - I(Y_1;Y_2) - \sum_{i=1,2} H (Y_i|X) \\
    &=& I(X;Y_1) + I(X_1,Y_2) - I(Y_1,Y-2) \\
    &=& 2I(X_i,Y_i) - I(Y_1,Y_2)
\end{eqnarray*}

\textbf{(b)} Conclude the channel $X\rightarrow\rightarrow(Y_1,Y_2)$ is less thatn twice the capacity of the channel $X\rightarrow\rightarrow(Y_1)$.

The capacity of a single-look channel $X\rightarrow Y_1$ is $C_1 = \max_{p(x)} I(X;Y_1)$.

The capacity of the channel $X \rightarrow (Y_1,Y_2)$ is 
\begin{eqnarray*}
    C_2 &=& \max_{p(x)}I(X;Y_1,Y_2) \\
    &=& \max_{p(x)} 2I(X;Y_1) -I(Y_1; Y_2) \\
    &\le& \max_{p(x)}2I(X;Y_1) = 2C_1;
\end{eqnarray*}
So two independent looks can do more than twice as good as single look.



\end{multicols}
\end{tiny}
\end{document}
