\documentclass[5pt]{article}

\usepackage{esdiff}
\usepackage{multirow}
\usepackage[large]{caption}


\newcommand{\secref}[1]{Section~\ref{sec:#1}}
\newcommand{\tabref}[1]{Table~\ref{tab:#1}}
\newcommand{\figref}[1]{Figure~\ref{fig:#1}}
\newcommand{\eqref}[1]{Equation~\ref{eq:#1}}

\textwidth=6.5in
\textheight=9in
\oddsidemargin=0in
\evensidemargin=0in
\topmargin=-.75in
\topskip=0in
\headheight=0in
\headsep=0in

\makeatletter
\def\@maketitle{%
  \newpage
  \begin{center}%
  \let \footnote \thanks
    {\LARGE \@title \par}%
    \vskip 1.5em%
    {\large
      \lineskip .5em%
      \begin{tabular}[t]{c}%�
        \@author
      \end{tabular}\par}%
    \vskip 1em%
    {\large \@date}%
  \end{center}%
  \par
  \vskip 1.5em}
\makeatother



\author{Michael P. Cutter}
\title{{\LaTeX} math mode exercise, BME 200}
\date{September 20, 2005}

\begin{document}
\maketitle
\[H(b) - H(a) \]
\[\sum_t a_i \log (a_i) \]
\[ - \sum_j \sum_i a_iP_{ij} \log(\sum_k a_kP_{kj})+\sum_t a_i \log (a_i) \]	
\[ \sum_{ij}a_iP_{ij} \log(\frac{ a_i }{\sum_{k} a_kP_{kj}}) (cite mult by 1)\]	
\[ \ge (\sum_{ij}a_iP_{ij}) \log(\frac{\sum_{ij} a_i }{\sum_{ij} b_j }) (cite log sum)\]	

b) If the matrix is doubly stochastic, the substituting $u_i$ = $1\over{m}$, we can easily see it satisfies $u=uP$
c) if the uniform is a stationary distribution, then $\frac{1}{m} = u_i = \sum_{j}u_{j}P_{ij}$ = $ \frac{1}{m}\sum_jP{ji}$ or $\sum_jP{ji}=1$ or that the matrix is doubly stochastic.


\section*{monotonicity of entropy per element}
\[ (chainRule)  \frac{H(X_1,X_2,\cdots,X_n)}{n} = \frac{\sum_{i=1}^n H(X_i | X^{i-1})}{n} \]
\[ \frac{(H(X_n | X^{n-1}) + \sum_{i=1}^{n-1} H(X_i |X^{i-1}) } {n} \]
\[ (cite??) \frac{H(X_n | X^{n-1}) + H(X_1,X_2,\cdots,X_{n-1}) } {n} \]
From stationarity it follows that for all $1\le i \le n$ $->$ $H(X_n | X^{n-1}) \le H(X_i | X^{i-1}) $
which further implies, by averaging both sides, that,
\[ H(X_n | X^{n-1}) \le \frac{ \sum_{i=1}^{n-1} H(X_i | X^{i-1})} {n-1} \]
\[ \frac{ H(X_i,X_2,\cdots,X_{n-1})} {n-1} \]
combining past two results yeilds, 
\[ \frac{H(X_1,X_2,\cdots,X_n)}{n} \le \frac{1}{n} [ \frac{ H(X_1,X_2,\cdots,X_n) } {n-1} + H(X_1,X_2,\cdots,X_{n-1}) ]  = \frac{H(X_1,X_2,\cdots,X_{n-1})}{n-1} \]
\\
b) By stationarity we have $\forall 1 \le i \le n$,
\[ H(X_n | X^{n-1} ) \le H(X_i | X^{i-1}) (whichimplies)\]
\[H(X_n | X^{n-1} = \frac{\sum_{i=1}^n H(X_n | X^{n-1} } {n} \]
\[ \frac{\sum_{i=1}^n H(X_i | X^{i-1} } {n} = \frac{H(X_1,X_2,\cdots,X_n)}{n}\]

\section*{Entropy rates of Markov chains}
[THIS PROBLEM NEEDS TABLES ADDED and questions!]
a) The stationary distribution is: $u_0=\frac{p_{10}}{p_{01}+p_{10}}$, $u_1=\frac{p_{01}}{p_{01}+p_{10}}$
Therefore the entropy is: \( H(X_2 \mid X_1) = u_0H(p_{01})+u_1H(p_{10}) = \frac{p_{10}H(p_{01})+ p_{01}H(p_{10})}{p_{01}+p_{10}}. \)

b) The entropy rate is at most 1 bit because process has only two states. This rate can be achieved iff \(p_{01}=p_{10}=1/2 \), in which case the process is i.i.d with \( \Pr(X_i = 0) =\Pr(X_i=1) = 1/2 \)

c) special case if the general two-state Markov chain, the entropy rate is \( H(X_2 | X_1)=u_0H(p)+u_1H(1)=\frac{H(p)}{p+1} = \frac{-p\log(p)-(1-p)\log(1-p)}{1+p} \)
\section*{problem 9}
Solution: initial conditions. For a Markov chain, by the data processing theorem we have 
$I(X_0  \: X_{n-1}) \ge I(X_0 \ : X_n)$

Therefore $H(X_0) - H(X_0 | X_{n-1}) \ge H(X_0) - H(X_0 | X_n)$ or $H(X_0 | X_n)$  increases with n

\section*{prob 11}
Solution: stationary processes A) $H(x_n | X_0) = H(X_{-n} | X_0).$ True cause: $H(X_n | X_0) = H(X_n,X_0) - H(X_0)$ and $H(X_{-n} | X_0) = H(X_{-n},X_0) - H(X_0)$ 

\end{document}
