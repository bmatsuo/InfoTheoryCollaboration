%%%%%%%%%%%%%%%%%%%%%%%%%%%%%%%%%%%%%%%%%%%%%%%%%%%%%%%%%%%%%%%%%%%%%%%%
%%        Annotated example of a homework submission in LaTeX         %%
%%               (note: LaTeX comments begin with a %)                %%
%%%%%%%%%%%%%%%%%%%%%%%%%%%%%%%%%%%%%%%%%%%%%%%%%%%%%%%%%%%%%%%%%%%%%%%%

%% Build this document with
%%
%%   pdflatex annotated_homework_example.tex
%%
%% or
%%
%%   latex annotated_homework_example.tex
%%   dvips annotated_homework_example.dvi
%%   ps2pdf annotated_homework_example.ps
%%
%% The result will be annotated_homework_example.pdf.
%%
%% IMPORTANT NOTE: Because this sample uses cross references, you must
%% build it *TWICE*. LaTeX finds all of the cross reference TARGETS first
%% and then, on the second run, plugs them all in to the appropriate
%% places.
%%
%% (note: any good LaTeX-friendly text editor will have these build
%%  lines built into it)

%%%%%%%%%%%%%%%%%%%%%%%%%%%%%%% PREAMBLE %%%%%%%%%%%%%%%%%%%%%%%%%%%%%%%

%% Every LaTeX document starts with a \documentclass line. The
%% document class sets up the most prominent features of the document
%% style (e.g., page styles, font, etc.).
%%
%% Standard document classes include article, book, and report.
%%
%%   *) The article document class is the most commonly used for short
%%      documents. It provides sectioning commands and flexible titling
%%      options.
%%
%%   *) The report document class gives you chapters and puts the title
%%      information centered on a separate unnumbered page.
%%
%%   *) The book document class gives you the ability to make parts and
%%      chapters and is usually used in two-sided printing mode (e.g.,
%%      page numbering alternates from being on the top left to top
%%      right of the page rather than in the bottom center).
%%
%% More advanced document classes include memoir, which is a drop-in
%% replacement for the standard classes that has many other features as
%% well. It folds in a rich feature set from some of the most popular
%% packages (packages are explained below).
%%
%% Optional arguments to LaTeX macros come in square brackets. In this
%% case, [10pt] tells the article class that we want 10 point font.
\documentclass[10pt]{article}

%% The default margins for article may seem large to you. They are
%% designed to look pleasant on a page. To use 1 inch margins instead,
%% call the geometry package with this line:
\usepackage[margin=1in]{geometry}

%% The enumitem package provides nicer list support to LaTeX.
\usepackage[shortlabels]{enumitem}

%% This setcounter line makes it so that \section commands do not have
%% numbers printed in front of them. Delete this line ENTIRELY *OR*
%% change the 0 to 1 or higher to make it so that \section commands
%% automatically get numbered.
%%
%% If you do increase the 0 to a positive integer or remove this line
%% entirely, later you can use \section* to get a single unnumbered
%% section.
\setcounter{secnumdepth}{0}

% The graphicx package lets us include images with \includegraphics
\usepackage{graphicx}

% The caption package lets us customize captions when we want to make
% formal figures and tables
\usepackage{caption}

%% Here, we use the fancyhdr and lastpage packages to customize the
%% header and footer of each page
\usepackage{fancyhdr, lastpage}
\usepackage{amsmath, amsthm, amssymb}
\pagestyle{fancy}
\fancyhf{}
%
%% Put your name and homework identification here
\lhead{John St. John}
\chead{EE~253~--- \today}
\rhead{Homework 1}
%
\cfoot{Page \thepage{} of \protect\pageref*{LastPage}}

%% These next two lines are only used if you are going to cross
%% reference equations in your submissions. They can probably be removed
%% for 481 submissions.
\usepackage{varioref}
\labelformat{equation}{(#1)}

% The hyperref package automatically links cross references to their
% targets and does LOTS more. In our case, we include it so that we can
% use \autoref. \autoref lets us do \autoref{eq:one} rather than
% Equation~\ref{eq:one}.
%
% We use the colorlinks option here, which colors hyperlinks rather
% than putting boxes around them. We then use the linkcolor option to
% make them blue rather than the default red. We could have used
% [pdfborder={0 0 0}] instead, which would leave the links uncolored,
% but it would remove the boxes.
%
% \usepackage{hyperref} must almost always be LAST \usepackage in
% preamble. Otherwise, you may get strange compilation errors!
\usepackage[colorlinks,linkcolor=blue]{hyperref}

%%%%%%%%%%%%%%%%%%%%%%%% MAIN DOCUMENT CONTENT %%%%%%%%%%%%%%%%%%%%%%%%%

% We surround our main content with a ``document'' ``environment.'' This
% marks the end of our LaTeX setup and the start of our document output.
%
% All ``environments'' are identified by \begin{...} and \end{...}
% macros.
\begin{document}

% This is a note about running LaTeX twice. It's centered, 3/4 the width
% of the printed type block, and it has a box around it. You can remove
% it.
%\centerline{\fbox{\parbox{0.75\columnwidth}{\textbf{SPECIAL NOTE:} If
%you notice a few ``\textbf{??}'' in funny places in your document (e.g.,
%the footer may say ``\textbf{Page 1 of ??}''), then run \LaTeX{} again.
%That information was not available in the first pass, but it will be
%filled in properly in the second pass.}}}

% Creates a section header with ``Manually Numbered Answers'' in it
\section{Chapter 2 Problems}

%%
%% Chapter 2, problems 2, 4, 10, 12, 15, 26, and 32.
%%

% Creates a list
\begin{itemize}

        % Pass optional items in square brackets to each \item macro.
        % Those optional items get displayed as the label for that item.
        \item[2.2.] Find a general inequality for $H(X)$ vs $H(Y)$.
                %
                % An enumerate list will automatically count each item
                % for you. Pass it the type of label you want (e.g., 1.
                % or a) or i) or ...). See below for another example.
                          let $y=g(x)$. $p(y)=\sum_{x:y=g(x)}p(x)$ Meaning sum over all $x$ given that $y=g(x)$. It is pretty clear that if there is a 1-1 mapping then $p(x)=p(y)$. 
                          
                          Consider a set of $x$ that maps to a single $y$. For this set $\sum_{x:y=g(x)}p(x)\ log\ p(x) \leq \sum_{x:y=g(x)}p(x)\ log\ p(y) = p(y)\ log\ p(y)$ since $p(y)\geq p(x)$. Now extending the argument to the entire range of $X$ and $Y$ we obtain:
          \begin{eqnarray*}
          p(x)\leq \sum_{x:y=g(x)}p(x)=p(y)\\
          H(X)=-\sum_{x}p(x)\ log\ p(x)\\
          =-\sum_{y}\sum_{x:y=g(x)}p(x)\ log\ p(x)\\
          \geq -\sum_{y}p(y)\ log\ p(y)\\
          =H(Y)
          \end{eqnarray*}
          \begin{enumerate}[a)]

          \item $H(X)$ vs $H(Y=2^{X})$\\
           $Y$ and $X$ are 1-1 so we have the special case of equality.

          \item $H(X)$ vs $H(Y=cos(X))$ \\
            There are multiple values of $X$ that could map to single values of $Y$ with this function. Thus we can only guarentee the general case where $H(X) \geq H(Y)$
          \end{enumerate}

        \item[2.4.] Show that the entropy of a function of $X$
                is $\leq$ than the entropy of $X$ by justifying the
                following steps.
                \begin{enumerate}[a)]
                  \item $H(X,g(X))=H(X) + H(g(X) | X)$\\
                    This is justified by the chain rule.
                  \item $=H(X)$ \\
                    $H(g(X)|X)=0$ because $g(X)$ is known for any value of $X$. 
                    Thus $H(g(X)|X)=\sum_{x}p(x)\ H(g(X)|X=x)=\sum_{x}0=0$ leaving
                    $H(X)$
                  \item $H(X,g(X))=H(g(X))+H(X|g(X))$ \\
                    Just a different application of the chain rule.
                  \item $\geq H(g(X))$\\
                   $H(X|g(X))\geq0$ and is equal to 0 when $g(X)$ gives an 
                   unambiguous mapping to one $X$. Otherwise it will be 
                   greater than 0. Thus $H(g(X))\leq H(X,g(X))$ 
                \end{enumerate}

        \item[2.10.] We have a set
        \[
         X=\begin{cases} X_1 & \mbox{with probability } \alpha \\
         X_2 & \mbox{with probability }(1-\alpha) \end{cases} 
        \]
         
          \begin{enumerate}[a)]
            \item What is $H(X)$ in term of $H(X_1), H(X_2), \alpha$? 
              \begin{eqnarray*} 
              	\theta = f(X) = \begin{cases}1 & \mbox{when } X=X_{1}\\ 2 & \mbox{when } X = X_{2} \end{cases}\\
	H(X) = H(X,f(X)) = H(\theta) + H(X|\theta)\\
	= H(\theta)+p(\theta = 1)\ H(X|\theta=1)+p(\theta=2)\ H(X|\theta=2)\\
	= H(\alpha) + \alpha\ H(X_{1}) + (1-\alpha)\ H(X_{2})
              \end{eqnarray*}
              where \(H(\alpha) = -\alpha\ log\ \alpha - (1 - \alpha )\ log\ (1-\alpha)\)
             
            %\item {\bf TODO!!!}
          \end{enumerate}
          
          
          
        \item[2.12.] 
       We are given that  \( p(x,y)=\left[ \frac{1}{3},\frac{1}{3},0,\frac{1}{3} \right] \) Find the following:
       
          \begin{enumerate}[a)]
            \item  H(X), H(Y)
              \begin{eqnarray*}
                H(X) = H(\frac{2}{3},\frac{1}{3})\\
                = \frac{2}{3} log(\frac{3}{2}) + \frac{1}{3}
                log(3)\\
                \approx 0.9183 bits \\
                H(Y) =  H(\frac{2}{3},\frac{1}{3})\\
                \approx 0.9183 bits
              \end{eqnarray*}
            \item H(X|Y), H(Y|X)
              \begin{eqnarray*}
                H(X|Y) = \frac{1}{3}H(X|Y=0) + \frac{2}{3}H(X|Y=1)\\
                =\frac{1}{3}H(\frac{1}{2},\frac {1}{2})+\frac{2}{3}H(0,1)\\
                =\frac{1}{3}1+\frac{2}{3}0 \\
                = \frac{1}{3} bits \\
                H(Y|X) = \frac{2}{3}H(Y|X=0)+\frac{1}{3}H(Y|X=1)\\
                =\frac{2}{3}H(\frac{1}{2},\frac{1}{2}) + \frac{1}{3}H(0,1)\\
                =\frac{2}{3} bits
              \end{eqnarray*}
            \item H(X,Y)
           	 \begin{eqnarray*}
            		H(X,Y)=H(X)+H(Y|X)\\
			\approx 1.585 bits
          	  \end{eqnarray*}
            \item H(Y) - H(Y|X)
            	\begin{eqnarray*}
			H(Y) - H(Y|X)\\ \approx 0.251 bits
		\end{eqnarray*}
            	
            \item I(X;Y)
            	\begin{eqnarray*}
			I(X;Y)=H(Y) - H(Y|X)\\ \approx 0.251 bits
		\end{eqnarray*}
            \item See~\autoref{fig:212venn} which shows the above equations on a Venn diagram.
            
            	 \begin{figure}[htbp]
			\begin{center}
				\includegraphics[scale=0.4]{Figs/venn_2_12-eps-converted-to.pdf}
				\caption{{\bf 2.12 Part e Venn Diagram}}
				\label{fig:212venn}
			\end{center}
		\end{figure}

          \end{enumerate}
          
        \item[2.15.] Markov chains: $X_{1}\rightarrow X_{2}\rightarrow \cdots \rightarrow X_{n}$, 
        		have joint probabilities of the following form: 
		\begin{eqnarray*}
			p(x_{1},x_{2},\ldots,x_{n})=p(x_{1})p(x_{2}|x_{1})\cdots p(x_{n}|x_{n-1})
		\end{eqnarray*}
		This means that for mutual information we can do some nice reduction.
		\begin{eqnarray*}
			I(X_{1};X_{2},\ldots,X_{n}) = \\
			let~~X_{3},\ldots,X_{n} = Y\\
			I(X_{1};X_{2},Y) = I(X_{1};X_{2})+I(X_{1};Y|X_{2})
		\end{eqnarray*}
		Now since this is a markov chain, and we know that $X_{1}$ and $Y$ are completely independent
		we can say that $I(X_{1};Y|X_{2})=0$ since $Y|X_{2}$ is just a subset of $Y$ which as noted
		previously is entirely independent of $X_{1}$. Thus the whole equation simplifies to $I(X_{1};X_{2})$

        \item[2.26.]
        		\begin{enumerate}[a)]
			\item $ln(x) \leq x-1$ for $0\leq x \leq \infty$
				As $x$ approaches 0 $x-1$ approaches -1 while $ln(x)$ approaches $-\infty$. $x-1$ 
				grows linearly while $ln(x)$ grows much slower, logarithmically, so for large numbers
				it is given that $x-1$ is larger.\\
				From the derivative of $x-1$ and $ln(x)$ we know that $x-1$ has a slope of $1$ for all 
				$x$, while $ln(x)$ has slope $\frac{1}{x}$. This means that when $x<1, ln(x)$ has an
				increasingly steep negative sloap, more steeply negative than 1, and when $x>1$ 
				it has an increasingly shallow 
				positive slope. The only point at which their slopes are equal is $x=1$, and since
				$ln(1)=0=1-1$ we can say that $(x-1)\geq ln(x)$. 
			\item
				Since $D(p||q)=\sum_{x}p(x)ln\left(\frac{p(x)}{q(x)}\right)$ that means that
				$-D(p||q)=-\sum_{x}p(x)ln\left(\frac{p(x)}{q(x)}\right)$ which is equal to the
				reciprocal of the log, giving us: $\sum_{x}p(x)ln\left(\frac{q(x)}{p(x)}\right)$\\
				Let $\frac{q(x)}{p(x)}$ be $x$, as shown before $x-1\geq ln(x)$ which 
				justifies the second step.\\
				The final step is fairly straight forward as well. In the equation
				$\sum_{x}p(x)\left(\frac{q(x)}{p(x)}-1\right)$ the $p(x)$ terms cancel
				leaving $\sum_{x}q(x)-1$. Since $q(x)$ is a probability, we know
				that over all $x$ it will sum to 1 meaning that the equation simplifies
				to 0.
			\item
				To equal 0, I have just shown that $\sum_{x}p(x)\left(\frac{q(x)}{p(x)}-1\right)=0$
				additionally, in my argument for part a, I showed that $ln(x) = x-1$ only when
				$x=1$. Thus when $\frac{q(x)}{p(x)}=1$ (when $q(x)=p(x)$) the value of $D$ 
				will be 0.
		\end{enumerate}
          
        \item[2.32.] See attached sheet.

\end{itemize}


\end{document}

%%%%%%%%%%%%%%%%%%%%%%%%%%%%% END DOCUMENT %%%%%%%%%%%%%%%%%%%%%%%%%%%%%