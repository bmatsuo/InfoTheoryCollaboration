%%%%%%%%%%%%%%%%%%%%%%%%%%%%%%%%%%%%%%%%%%%%%%%%%%%%%%%%%%%%%%%%%%%%%%%%
%%        Annotated example of a homework submission in LaTeX         %%
%%               (note: LaTeX comments begin with a %)                %%
%%%%%%%%%%%%%%%%%%%%%%%%%%%%%%%%%%%%%%%%%%%%%%%%%%%%%%%%%%%%%%%%%%%%%%%%

%% Build this document with
%%
%%   pdflatex annotated_homework_example.tex
%%
%% or
%%
%%   latex annotated_homework_example.tex
%%   dvips annotated_homework_example.dvi
%%   ps2pdf annotated_homework_example.ps
%%
%% The result will be annotated_homework_example.pdf.
%%
%% IMPORTANT NOTE: Because this sample uses cross references, you must
%% build it *TWICE*. LaTeX finds all of the cross reference TARGETS first
%% and then, on the second run, plugs them all in to the appropriate
%% places.
%%
%% (note: any good LaTeX-friendly text editor will have these build
%%  lines built into it)

%%%%%%%%%%%%%%%%%%%%%%%%%%%%%%% PREAMBLE %%%%%%%%%%%%%%%%%%%%%%%%%%%%%%%

%% Every LaTeX document starts with a \documentclass line. The
%% document class sets up the most prominent features of the document
%% style (e.g., page styles, font, etc.).
%%
%% Standard document classes include article, book, and report.
%%
%%   *) The article document class is the most commonly used for short
%%      documents. It provides sectioning commands and flexible titling
%%      options.
%%
%%   *) The report document class gives you chapters and puts the title
%%      information centered on a separate unnumbered page.
%%
%%   *) The book document class gives you the ability to make parts and
%%      chapters and is usually used in two-sided printing mode (e.g.,
%%      page numbering alternates from being on the top left to top
%%      right of the page rather than in the bottom center).
%%
%% More advanced document classes include memoir, which is a drop-in
%% replacement for the standard classes that has many other features as
%% well. It folds in a rich feature set from some of the most popular
%% packages (packages are explained below).
%%
%% Optional arguments to LaTeX macros come in square brackets. In this
%% case, [10pt] tells the article class that we want 10 point font.
\documentclass[9pt]{article}
\pagestyle{empty}

%% The default margins for article may seem large to you. They are
%% designed to look pleasant on a page. To use 1 inch margins instead,
%% call the geometry package with this line:
\usepackage{savetrees}


% The graphicx package lets us include images with \includegraphics
\usepackage{graphicx}
\usepackage{epstopdf}
% The caption package lets us customize captions when we want to make
% formal figures and tables
\usepackage{multicol}
\usepackage{amsmath, amsthm, amssymb}

%%%%%%%%%%%%%%%%%%%%%%%% MAIN DOCUMENT CONTENT %%%%%%%%%%%%%%%%%%%%%%%%%

% We surround our main content with a ``document'' ``environment.'' This
% marks the end of our LaTeX setup and the start of our document output.
%
% All ``environments'' are identified by \begin{...} and \end{...}
% macros.
\begin{document}
\begin{multicols}{2}
% This is a note about running LaTeX twice. It's centered, 3/4 the width
% of the printed type block, and it has a box around it. You can remove
% it.
%\centerline{\fbox{\parbox{0.75\columnwidth}{\textbf{SPECIAL NOTE:} If
%you notice a few ``\textbf{??}'' in funny places in your document (e.g.,
%the footer may say ``\textbf{Page 1 of ??}''), then run \LaTeX{} again.
%That information was not available in the first pass, but it will be
%filled in properly in the second pass.}}}

% Creates a section header with ``Manually Numbered Answers'' in it
\section{Chapter 2 Problems}

%%
%% Chapter 2, problems 2, 4, 10, 12, 15, 26, and 32.
%%

% Creates a list

        % Pass optional items in square brackets to each \item macro.
        % Those optional items get displayed as the label for that item.
\section* {Find a general inequality for $H(X)$ vs $H(f(X))$}
                %
                % An enumerate list will automatically count each item
                % for you. Pass it the type of label you want (e.g., 1.
                % or a) or i) or ...). See below for another example.
                          let $y=g(x)$. $p(y)=\sum_{x:y=g(x)}p(x)$ Meaning sum over all $x$ given that $y=g(x)$. It is pretty clear that if there is a 1-1 mapping then $p(x)=p(y)$. 
                          
                          Consider a set of $x$ that maps to a single $y$. For this set $\sum_{x:y=g(x)}p(x)\ log\ p(x) \leq \sum_{x:y=g(x)}p(x)\ log\ p(y) = p(y)\ log\ p(y)$ since $p(y)\geq p(x)$. Now extending the argument to the entire range of $X$ and $Y$ we obtain:
          \(
          p(x)\leq \sum_{x:y=g(x)}p(x)=p(y)\) and \(
          H(X)=-\sum_{x}p(x)\ log\ p(x)
          =-\sum_{y}\sum_{x:y=g(x)}p(x)\ log\ p(x)
          \geq -\sum_{y}p(y)\ log\ p(y)
          =H(Y)
          \)

          {\bf a.)} $H(X)$ vs $H(Y=2^{X})$\\
           $Y$ and $X$ are 1-1 so we have the special case of equality.

          {\bf b.)} $H(X)$ vs $H(Y=cos(X))$ \\
            There are multiple values of $X$ that could map to single values of $Y$ with this function. Thus we can only guarentee the general case where $H(X) \geq H(Y)$

\section*{ Show that $H(f(X))\leq H(X)$}
                  {\bf a.)} $H(X,g(X))=H(X) + H(g(X) | X)$\\
                    This is justified by the chain rule.
                    
                  {\bf b.)} $=H(X)$ \\
                    $H(g(X)|X)=0$ because $g(X)$ is known for any value of $X$. 
                    Thus $H(g(X)|X)=\sum_{x}p(x)\ H(g(X)|X=x)=\sum_{x}0=0$ leaving
                    $H(X)$
                    
                  {\bf c.)} $H(X,g(X))=H(g(X))+H(X|g(X))$ \\
                    Just a different application of the chain rule.
                    
                  {\bf d.)} $\geq H(g(X))$\\
                   $H(X|g(X))\geq0$ and is equal to 0 when $g(X)$ gives an 
                   unambiguous mapping to one $X$. Otherwise it will be 
                   greater than 0. Thus $H(g(X))\leq H(X,g(X))$ 

\section*{ $H$ of disjoint mixture
        \(
         X=\begin{cases} X_1 & w/p\ \alpha \\
         X_2 & w/p\ (1-\alpha) \end{cases} 
        \)}
         
           {\bf a.)} What is $H(X)$ in term of $H(X_1), H(X_2), \alpha$? 
              \(
              	\theta = f(X) = \begin{cases}1 & when X=X_{1}\\ 2 & when X = X_{2} \end{cases}\) and \(
	H(X) = H(X,f(X)) = H(\theta) + H(X|\theta)\ 
	= H(\theta)+p(\theta = 1)\ H(X|\theta=1)+p(\theta=2)\ H(X|\theta=2)\ 
	= H(\alpha) + \alpha\ H(X_{1}) + (1-\alpha)\ H(X_{2})
              \)
              where \(H(\alpha) = -\alpha\ log\ \alpha - (1 - \alpha )\ log\ (1-\alpha)\)
             
            %\item {\bf TODO!!!}
                  
\section*{Application of $H$ and $I$}
       We are given that  \( p(x,y)=\left[ \frac{1}{3},\frac{1}{3};0,\frac{1}{3} \right] \) Find the following:
       
            {\bf a.)}Compute  \(H(X)\) and \( H(Y)\):~~
              \(
                H(X) = H(\frac{2}{3},\frac{1}{3})
                = \frac{2}{3} log(\frac{3}{2}) + \frac{1}{3}log(3)\\
                \approx 0.9183 bits \) and \(
                H(Y) =  H(\frac{2}{3},\frac{1}{3})\
                \approx 0.9183 bits
              \)
              
            {\bf b.)} Compute \(H(X|Y)\) and \(H(Y|X)\):~~
              \(
                H(X|Y) = \frac{1}{3}H(X|Y=0) + \frac{2}{3}H(X|Y=1)
                =\frac{1}{3}H(\frac{1}{2},\frac {1}{2})+\frac{2}{3}H(0,1)
                =\frac{1}{3}1+\frac{2}{3}0 
                = \frac{1}{3} bits \) 
                
                \(
                H(Y|X) = \frac{2}{3}H(Y|X=0)+\frac{1}{3}H(Y|X=1)
                =\frac{2}{3}H(\frac{1}{2},\frac{1}{2}) + \frac{1}{3}H(0,1)
                =\frac{2}{3} bits
              \)
              
            {\bf c.)}
           	 \(
            		H(X,Y)=H(X)+H(Y|X)
			\approx 1.585 bits
          	  \)
          	  
            {\bf d.)} \(H(Y) - H(Y|X) \approx 0.251 bits\)
            	
            {\bf e.)} \( I(X;Y) =H(Y) - H(Y|X) \approx 0.251 bits
		\)
		
            {\bf f.}  Imagine a two circle venn. $H(X,Y)$ represents the overlapped portion plus the two independent portions. $H(X)$ is one circle and $H(Y)$ is the other. The overlapped portion is represented by either $I(X;Y)$ or $H(Y)-H(Y|X)$.
          
\section*{Mutual information and Markov chain} Markov chains: $X_{1}\rightarrow X_{2}\rightarrow \cdots \rightarrow X_{n}$, 
        		have joint probabilities of the following form: 
		\(
			p(x_{1},x_{2},\ldots,x_{n})=p(x_{1})p(x_{2}|x_{1})\cdots p(x_{n}|x_{n-1})
		\)
		This means that for mutual information we can do some nice reduction.
		first let \( X_{3},\ldots,X_{n} = Y\) then you get \(
			I(X_{1};X_{2},Y) = I(X_{1};X_{2})+I(X_{1};Y|X_{2})  \)
		Now since this is a markov chain, and we know that $X_{1}$ and $Y$ are completely independent
		we can say that $I(X_{1};Y|X_{2})=0$ because $X_1$ and $Y$ are conditionally independent given $X_2$. 
		Thus the whole equation simplifies to $I(X_{1};X_{2})$

\section*{Using concavity to prove stuff}
			{\bf a.)} $ln(x) \leq x-1$ for $0\leq x \leq \infty$\\
				If the second derivative of a function is positive, that means that it is concave up, or convex. If it is always greater than 0, it will never not be convex. If it could be zero on the interval, then we can't make this claim, and the converse isn't necessarily true.
				Let \(f(x)=x-1-ln\ x \) then 
				\(
				 f'(x)= 1-\frac{1}{x}\) and \(
				 f''(x)=\frac{1}{x^2} > 0
				\)
				 so we can say that $f(x)$ is strictly convex and a local minimum is also a global minimum. To find the local minimum we just set $f'(x)=0$ and get $x=1$. So $f(x) \geq f(1)$ which means that $x-1-ln\ x \geq 1-1-ln\ 1 = 0$. Showing that $x-1 \geq ln\ x$ on the interval $(0,\infty)$.
				 
			{\bf b.)} Justify the following steps. 
				Let A be the set of $x$ such that $p(x) > 0$
				\(
				-D_e(p||q)=\sum_{x\in A} p(x) ln \frac{q(x)}{p(x)}
				\leq \sum_{x\in A} p(x)\left( \frac{q(x)}{p(x)}-1\right)
				=\sum_{x\in A} q(x) - \sum_{x\in A} p(x)
				\leq 0
				\)
				The first step follows from the definition of D, the second step follows from the inequality $ln\ t \leq t - 1$, the third from expanding the sum, and the last step from the fact that the $q(A) \leq 1$ and $p(A)=1$.
				
			{\bf c.)} Conditions for equality?
			We have the inequality $ln\ t \leq t-1$ from a previous problem, and showed equality iff $t=1$. Therefore we have equality if $\frac{q(x)}{p(x)}=1$ for all $x\in A$. This implies that $p(x)=q(x)$ for all $x$ and then we'll have equality in the last step as well. 	
          
\section*{Calculating and estimating error}
        \(P(X,Y) = \left( \begin{array}{ccc}
\frac{1}{6} & \frac{1}{12} & \frac{1}{12} \\
\frac{1}{12} & \frac{1}{6} & \frac{1}{12} \\
\frac{1}{12} & \frac{1}{12} & \frac{1}{6} \end{array} \right)\)
 where $X$ is indexed $[a,b,c]$ and $Y$ is indexed $[1,2,3]$. Let $\hat{X}(Y)$  be an estimator for $X$ based on $Y$ and let $P_e=Pr\{\hat{X}(Y)\neq X\}$

        {\bf a.)}
        	From inspection we see that 
        	\(\hat{X}(y)=\left\{ \begin{array}{cc}
        		1 & y=a\\
        		2 & y=b\\
        		3 & y=c \end{array}\right. \)
	and the associated $P_e$ is the sum of 
	\[
	P(1,b),  P(1,c), P(2,a),
	 P(2,c), P(3,a), P(3,b) \] Therefore $P_e=\frac{1}{2}$
        
        {\bf b.)}
        	From Fano's inequality we know \(P_e \geq \frac{H(X|Y)-1}{log\ |\chi|}\). \(
        	H(X|Y)= H(X|Y=a)Pr\{y=a\}+H(X|Y=b)Pr\{y=b\}+H(X|Y=c)Pr\{y=c\}
        	=H\left(\frac{1}{2},\frac{1}{4},\frac{1}{4}\right)Pr\{y=a\}+H\left(\frac{1}{2},\frac{1}{4},\frac{1}{4}\right)Pr\{y=b\}+H\left(\frac{1}{2},\frac{1}{4},\frac{1}{4}\right)Pr\{y=c\}
        	=H\left(\frac{1}{2},\frac{1}{4},\frac{1}{4}\right)\left( Pr\{y=a\}+Pr\{y=b\}+Pr\{y=c\}\right)
        	=H\left(\frac{1}{2},\frac{1}{4},\frac{1}{4}\right)
        	=1.5
        	\) bits. 
        	Our estimator of error $P_e\geq \frac{1.5-1}{log\ 3}=0.316$ is not very close to Fano's bound in this form. Since $\hat{X}\in \chi$ we can use the stronger form of Fano's inequality 
        	\( P_e \geq \frac{H(X|Y)-1}{log(|\chi|-1)}  \) to get \( P_e \geq \frac{1.5-1}{log\ 2}=\frac{1}{2} \) which is a great bound on our actual $P_e$.

\end{multicols}

\end{document}

%%%%%%%%%%%%%%%%%%%%%%%%%%%%% END DOCUMENT %%%%%%%%%%%%%%%%%%%%%%%%%%%%%