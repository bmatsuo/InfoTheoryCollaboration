\documentclass[10pt]{article}

\usepackage{fullpage,latexsym,amsthm,amsmath,amsfonts,algorithmic,graphicx,subfigure,color,verbatim,epsfig,fancyhdr}
\pagestyle{empty}
% Define new commands.
\DeclareMathOperator*{\E}{\mbox{\text{E}}}

\newcommand{\pref}[1]{{(\ref{#1})}}
\newcommand{\bmat}[1]{{\color{red}{#1}}}

\newtheorem{defn}{Definition}[section]
\newtheorem{lemma}{Lemma}[section]
\newtheorem{corollary}{Corollary}[section]
\newtheorem{question}{Question}[section]
\newtheorem{proposition}{Proposition}[section]
\usepackage{multicol}
\usepackage{savetrees}


% Begin the actual document content here.
\begin{document}
\begin{scriptsize}
\begin{multicols}{3}

\section*{Entropy, Relative Entropy and Mutual Information}
Definitions, alternate expressions, and properties.
\begin{align}
\underbrace{H(X)}_\text{entropy}&= -\sum_{x\in X} p(x) \log\ p(x) \label{eq:entropy}\\
H(X)& \geq 0\\
H_b(X)&= (\log_b a)H_a(X)\\
H(X|Y)& \leq  H(X)\text{ eq iff ind} \label{eq:conditioning}\\
X^n&=[X_1,\ldots,X_n]\\
H(X^n)& \leq \sum_{i=1}^n H(X_i)\text{ eq iff ind} \label{eq:jointsum}\\
H(X)&\leq  \log |\chi|  \text{ eq iff iid}\label{eq:alphabetentropy}\\
H(p)&\text{ is concave in $p$}\\
\underbrace{D(p||q)}_\text{rel ent} & = \sum_x p(x) \log \frac{p(x)}{q(x)}\\
\underbrace{I(X;Y)}_\text{mut inf} & = \sum_{x\in X} \sum_{y \in Y} p(x,y) \log \frac{p(x,y)}{p(x)p(y)}\\
H(X)&=E_p \log \frac{1}{p(X)}\\
H(X,Y)&=E_p\log \frac{1}{p(x,y)}\\
H(X|Y)&=E_p\log \frac{p(X,Y)}{p(X)p(Y)}\\
H(X,Y|Z) &= H(X|Z) + H(Y|X,Z) \\
D(p||q)&=E_p\log \frac{p(X)}{q(X)}\\
I(X;Y)&=H(X)-H(X|Y)\\
&=H(Y)-H(Y|X)\\
&=H(X)+H(Y)-H(X,Y)\\
D(p||q)&\geq 0 \text{ eq iff $p=q$}\\
I(X;Y)&=D(p(x,y)||p(x)p(y))\\
	&\geq 0 \text{ eq iff $p(x,y)=p(x)p(y)$}
\end{align}
Given $|\chi|=m$ and $u$ is the uniform distribution over $\chi$, we get equation~\ref{eq:chimu}.
\begin{align}
D(p||u)&=\log m - H(p)\label{eq:chimu}\\
D(p||q)&\text{ is convex in the pair }(p,q)
\end{align}
Following are chain rules:
\begin{align}
H(X^n)&=\sum_{i=1}^n H(X_i|X^{i-1})\\
I(X^n;Y)&=\sum_{i=1}^n I(X_i;Y|X^{i-1})\\
D(p(x,y)||q(x,y))&=D(p(x)||q(x))\\
&+D(p(y|x)||q(y|x))
\end{align}
{\bf Jensen's Inequality:} if $f$ is a convex function, then
\begin{equation}
Ef(x) \geq f(EX)
\end{equation}
{\bf Log sum inequality:} for n positive numbers $a_1,\ldots,a_n$ and $b_1,\ldots,b_n$,
\begin{equation}
\sum_{i=1}^n a_i \log \frac{a_i}{b_i} \geq \left( \sum_{i=1}^n a_i \right) \log \frac{\sum_{i=1}^n a_i}{\sum_{i=1}^n b_i}
\end{equation}
with equality only if $\frac{a_i}{b_i}$ is constant. 

{\bf Data processing inequality:} given some Markov processes $X\rightarrow Y\rightarrow Z$ we know $I(X;Y)\geq I(X;Z)$

{\bf Sufficient Statistic:} $T(X)$ is sufficient relative to $\{f_\theta (x)\}$ iff $I(\theta;X)=I(\theta;T(X))$ for all distributions on $\theta$.

{\bf Fano's inequality:} Let $P_e=Pr\{\hat{X}(Y)\neq X\}$. Then $H(P_e)+P_e \log |\chi| \geq H(X|Y)$.

{\bf Inequality.} If $X$ and $X'$ are independent and identically distributed then $Pr(X=X') \geq 2^{-H(X)}$


\section*{Asymptotic Equipartition Property (AEP)}
``Almost all events are almost equally surprising''. Specifically if $X_1,\ldots,X_n$ are iid with respect to $p(x)$ then :
\begin{equation}
-\frac{1}{n}\log p(X_1,\ldots,X_n)\rightarrow H(X) \text{in probability}
\end{equation}

A typical set $A_\epsilon^{(n)}$ is the set of sequences $x_1,\ldots, x_n$ that follow:
\begin{equation}
2^{-n(H(X)+\epsilon)}\leq p(X^n)\leq 2^{-n(H(X)-\epsilon)}
\end{equation}

Some properties of a typical set:
\begin{enumerate}
\item if $(x_1,\ldots,x_n) \in A_\epsilon^{(n)}$ then $p(x_1,\ldots,x_n) = 2^{-n(H\pm \epsilon)}$
\item Pr$\left\{A_\epsilon^{(n)}\right\}>1-\epsilon$ for sufficiently large $\epsilon$
\item $|A_\epsilon^{(n)}| \leq 2^{n(H(X)+\epsilon}$
\end{enumerate}

Definition of $a_n \doteq b_n$ means that $\frac{1}{n}\log \frac{a_n}{b_n} \rightarrow 0$ as $n\rightarrow \infty$.

Smallest probable set: Let $X_1,\ldots, X_n$ be iid with respect to $p(x)$ and for $\gamma < \frac{1}{2}$, let $B_\gamma^{(n)} \subset \chi^n$ be the smallest set such that $Pr\{ B_\gamma^{(n)} \} \geq 1-\gamma $ Then we have $|B_\gamma^{(n)}| \doteq 2^{nH}$

\section*{Other useful laws and properties}
{\bf General Statistical things} Expected value $E(x)=\sum_{i=1}^n x_i p(x_i)$

Variance: $\sigma^2 = E\left[ \left(X-\mu\right)^2\right]$

{\bf Law of Large Numbers} Weak law:
\begin{eqnarray*}
\bar{X}_n\overset{p}\rightarrow \mu \text{ when } n \rightarrow \infty\\
\lim_{n\rightarrow \infty} Pr\left(|\bar{X}_n-\mu| < \epsilon \right) = 1
\end{eqnarray*}
Strong law:
\begin{equation}
Pr\left(\lim_{n\rightarrow \infty} \bar{X}_n = \mu \right) = 1
\end{equation}

{\bf Derivation Rules} Chain rule: $(f \circ g)'(x)=f'(g(x))g'(x)$

Product Rule: $(f\cdot g)' = f' \cdot g + f \cdot g'$

Quotient Rule: \[f(x)=\frac{g(x)}{h(x)}\text{ and }f'(x)=\frac{g'(x)h(x)-g(x)h'(x)}{\left(h(x)\right)^2}\]


\section*{Entropy Rate}
\begin{equation}
H(\chi) = lim_{n \to \infty} \frac{1}{n} H(X^n) \\
\end{equation}
\begin{equation}
H^\prime(\chi) = lim_{n \to \infty} H(X_n | X^{n-1})
\end{equation}

\noindent If stationary and stochastic:
\begin{equation}
H(\chi) = H^\prime(\chi)
\end{equation}

\noindent Entropy rate of stationary Markov chain (transition matrix $P$) :
\begin{equation}
H(\chi) = H(X_2|X_1) = -\sum_{i,j} \mu_i P_{ij}log(P_ij)
\end{equation}

\noindent{\bf Stationarity } if a distribution at time $n+1$ is the same as time $n$, it is called a {\it stationary distribution}. If the initial state of Markov chain is a {\it stationary distribution} then the Markov chain is a stationary process. A finite-state Markov chain that is irreducible (positive prob. from any state to any other in finite steps) and aperiodic (largest factor of length of paths of state to itself is 1) has a unique stationary distribution that $lim_{n \to \infty}X_n$ converges to.

\noindent{\bf Functions of Markov Chain}
$X^n$ is stationary Markov chain, and $Y_i = \phi(X_i)$ then

\begin{eqnarray*}
H(Y_n | Y^{n-1}, X_1) \leq H(\gamma) \leq H(Y_n|Y^{n-1})\\
lim_{n \to \infty}H(Y_n|Y^{n-1},X_1) = H(\gamma) \\
= lim_{n \to \infty} H(Y_n | Y^{n-1})
\end{eqnarray*}

\noindent{\bf Ces\'{a}ro mean} If $a_n \to a$ and $b_n = \frac{1}{n}\sum_{i=1}^n a_i$ then $b_n \to a$. 

\noindent{\bf 2 state Markov Chain} given $P=[1-\alpha, \alpha; \beta, 1-\beta]$ the stationary distribution $u$ where $uP = u$, is $u_1 = \frac{\beta}{\alpha+\beta}$, $u_2 = \frac{\alpha}{\alpha+\beta}$, and $H(\chi) = \frac{\beta}{\alpha + \beta}H(\alpha) + \frac{\alpha}{\alpha+\beta}H(\beta)$. 

\noindent{\bf Doubly Stochastic} probability transition matrix $[P_{ij}] = Pr\{X_{n+1} = j | X_n = i\}$ is doubly stochastic if rows and cols sum to $1$. The uniform distribution is a stationary distribution iff $P$ is double stochastic.

\end{multicols}
\end{scriptsize}
\end{document}
