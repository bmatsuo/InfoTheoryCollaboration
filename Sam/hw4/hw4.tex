\documentclass{report}
% Change "report" to "article" to get a page number on title page
\usepackage{fitch}
\usepackage{amsmath,amsfonts,amsthm,amssymb}
\usepackage{setspace}
\usepackage{fancyhdr}
\usepackage{lastpage}
\usepackage{extramarks}
\usepackage{chngpage}
\usepackage{soul,color}
\usepackage{graphicx,float,wrapfig}

% Homework Specific Information
\newcommand{\hmwkTitle}{Homework 4}
\newcommand{\hmwkDueDate}{Tues., Nov. 16}
\newcommand{\hmwkClass}{EE253}
\newcommand{\hmwkClassTime}{}
\newcommand{\hmwkClassInstructor}{Prof. Sadjadpour}
\newcommand{\hmwkAuthorName}{Sam Wood}

% In case you need to adjust margins:
\topmargin=-0.45in      %
\evensidemargin=0in     %
\oddsidemargin=0in      %
\textwidth=6.5in        %
\textheight=9.0in       %
\headsep=0.25in         %

% Setup the header and footer
\pagestyle{fancy}                                                       %
\lhead{\hmwkAuthorName}                                                 %
\rhead{\firstxmark}                                                     %
\lfoot{\lastxmark}                                                      %
\cfoot{}                                                                %
\rfoot{Page\ \thepage\ of\ \protect\pageref{LastPage}}                          %
\renewcommand\headrulewidth{0.4pt}                                      %
\renewcommand\footrulewidth{0.4pt}                                      %

% This is used to trace down (pin point) problems
% in latexing a document:
%\tracingall

%%%%%%%%%%%%%%%%%%%%%%%%%%%%%%%%%%%%%%%%%%%%%%%%%%%%%%%%%%%%%
% Some tools
\newcommand{\enterProblemHeader}[1]{\nobreak\extramarks{#1}{#1 continued on next page\ldots}\nobreak%
                                    \nobreak\extramarks{#1 (continued)}{#1 continued on next page\ldots}\nobreak}%
\newcommand{\exitProblemHeader}[1]{\nobreak\extramarks{#1 (continued)}{#1 continued on next page\ldots}\nobreak%
                                   \nobreak\extramarks{#1}{}\nobreak}%

\newlength{\labelLength}
\newcommand{\labelAnswer}[2]
  {\settowidth{\labelLength}{#1}%
   \addtolength{\labelLength}{0.25in}%
   \changetext{}{-\labelLength}{}{}{}%
   \noindent\fbox{\begin{minipage}[c]{\columnwidth}#2\end{minipage}}%
   \marginpar{\fbox{#1}}%

   % We put the blank space above in order to make sure this
   % \marginpar gets correctly placed.
   \changetext{}{+\labelLength}{}{}{}}%

\newcommand{\homeworkProblemName}{}%
\newcounter{homeworkProblemCounter}%
\newenvironment{homeworkProblem}[1][Problem \arabic{homeworkProblemCounter}]%
  {\stepcounter{homeworkProblemCounter}%
   \renewcommand{\homeworkProblemName}{#1}%
   \section*{\homeworkProblemName}%
   \enterProblemHeader{\homeworkProblemName}}%
  {\exitProblemHeader{\homeworkProblemName}}%

\newcommand{\problemAnswer}[1]
  {\noindent\fbox{\begin{minipage}[c]{\columnwidth}#1\end{minipage}}}%

\newcommand{\problemLAnswer}[1]
  {\labelAnswer{\homeworkProblemName}{#1}}

\newcommand{\homeworkSectionName}{}%
\newlength{\homeworkSectionLabelLength}{}%
\newenvironment{homeworkSection}[1]%
  {% We put this space here to make sure we're not connected to the above.
   % Otherwise the changetext can do funny things to the other margin

   \renewcommand{\homeworkSectionName}{#1}%
   \settowidth{\homeworkSectionLabelLength}{\homeworkSectionName}%
   \addtolength{\homeworkSectionLabelLength}{0.25in}%
   \changetext{}{-\homeworkSectionLabelLength}{}{}{}%
   \subsection*{\homeworkSectionName}%
   \enterProblemHeader{\homeworkProblemName\ [\homeworkSectionName]}}%
  {\enterProblemHeader{\homeworkProblemName}%

   % We put the blank space above in order to make sure this margin
   % change doesn't happen too soon (otherwise \sectionAnswer's can
   % get ugly about their \marginpar placement.
   \changetext{}{+\homeworkSectionLabelLength}{}{}{}}%

\newcommand{\sectionAnswer}[1]
  {% We put this space here to make sure we're disconnected from the previous
   % passage

   \noindent\fbox{\begin{minipage}[c]{\columnwidth}#1\end{minipage}}%
   \enterProblemHeader{\homeworkProblemName}\exitProblemHeader{\homeworkProblemName}%
   \marginpar{\fbox{\homeworkSectionName}}%

   % We put the blank space above in order to make sure this
   % \marginpar gets correctly placed.
   }%

%%%%%%%%%%%%%%%%%%%%%%%%%%%%%%%%%%%%%%%%%%%%%%%%%%%%%%%%%%%%%

%%%%%%%%%%%%%%%%%%%%%%%%%%%%%%%%%%%%%%%%%%%%%%%%%%%%%%%%%%%%%
% Make title
\title{\vspace{2in}\textmd{\textbf{\hmwkClass:\ \hmwkTitle}}\\\normalsize\vspace{0.1in}\small{Due\ on\ \hmwkDueDate}\\\vspace{0.1in}\large{\textit{\hmwkClassInstructor\ \hmwkClassTime}}\vspace{3in}}
\date{}
\author{\textbf{\hmwkAuthorName}}
%%%%%%%%%%%%%%%%%%%%%%%%%%%%%%%%%%%%%%%%%%%%%%%%%%%%%%%%%%%%%
\begin{document}
\begin{spacing}{1.1}
\maketitle
\newpage

\begin{homeworkProblem}[Problem 7.3: Channels w/ Memory]
\begin{align*}
nC &= n\sum_{i=1}^n I(X_i;Y_i) = nH(X_1) - nH(X_1|Y_1) \leq nH(X_1) &\textrm{(no feedback), same distribution $p(x)$} \\ 
nH(X_1) &= H(X_1, X_2, ...,X_n) & \textrm{$X_i$s are i.i.d.} \\
&= H(X_1, X_2, ...,X_n) - H(X_1, X_2, ..., X_n | Y_1, Y_2, ..., Y_n, Z_1, Z_2, ..., Z_n)  & \textrm{new term is 0}\\
&= H(X_1, X_2, ...,X_n) - H(X_1, X_2, ..., X_n | Y_1, Y_2, ..., Y_n) & \textrm{$Z^n$ is independent of $X^n$} \\
&= I(X_1, X_2, ..., X_n; Y_1, Y_2, ..., Y_n) & \textrm{chain rule}\\
nC &\leq  I(X_1, X_2, ..., X_n; Y_1, Y_2, ..., Y_n) & \textrm{transitivity of inequalities}
%\sum_{i=1}^n H(X_i|Y_i) &\geq \sum_{i=1}^n H(X_i|Y_1,Y_2,...,Y_n) & \textrm{conditioning reduces entropy} \\
%\sum_{i=1}^n H(X_i) - H(X_i|Y_i) &\leq \sum_{i=1}^n H(X_i) - H(X_i|Y_1,Y_2,...,Y_n) &\textrm{math for positive $H(X_i)$} \\
%\sum_{i=1}^n I(X_i|Y_i)  &\leq \sum_{i=1}^n H(X_i) - H(X_i|Y_1,Y_2,...,Y_n) & \textrm{chain rule}\\
%nC  &\leq \sum_{i=1}^n H(X_i) - H(X_i|Y_1,Y_2,...,Y_n) &\textrm{definition of C, I(X,Y) independent of $i$} \\
%nC  &\leq \sum_{i=1}^n I(X_i;Y_1,Y_2,...,Y_n) & \textrm{chain rule} \\
%nC  &\leq \sum_{i=1}^n I(X_i;Y_1,Y_2,...,Y_n|X_{i-1},...,X_1) & \textrm{no feedback} \\
%nC  &\leq \max_{p(x)}I(X_1,X_2, ..., X_n;Y_1,Y_2,...,Y_n) & \textrm{chain rule} 
\end{align*}
\end{homeworkProblem} 

\begin{homeworkProblem}[Problem 7.4: Channel Capacity]
\begin{itemize}
\item[a.] \begin{align*} 
			C &=  \max_{p(x)}\{ I(X;Y)\} = \max_{p(x)}\{ H(Y) - H(Y|X)\} \\
			  &= \max_{p(x)}\{ H(Y) - \sum_{x \in \chi} p(x)H(Y|X=x)\} \\
			  &= \max_{p(x)}\{ H(Y) - H(\frac{1}{3}, \frac{1}{3}, \frac{1}{3})\} \\
		  \end{align*}
		Note that $C$ is maximized when $Y$ is uniform, thus $C = log(11) - log(3) = log(11/3)$.
\item[b.] $C$ is maximized when $Y$ is uniform. Additionally, if $X$ is uniform then $Y$ is uniform. Therefore, $p(x) = \frac{1}{11}$.
\end{itemize}
\end{homeworkProblem} 

\begin{homeworkProblem}[Problem 7.7: Cascade]
Given a binary symmetric channel with transition matrix:
\[
 A =
 \begin{pmatrix}
  1-p & p   \\
  p   & 1-p \\
 \end{pmatrix}
\] show the error probability of a cascade of $n$ independent binary symmetric channels. We note that we can diagonalize the matrix and then raise this matrix to the power of $n$ to find a simple expression for the error probability. Using $A$s eigenvectors as columns, we construct $P$ and $P^{-1}$:
\begin{align}
P&=\begin{pmatrix} 1 & -1 \\ 1 & 1 \end{pmatrix}, P^{-1} = \frac{1}{2}\begin{pmatrix} 1 & 1 \\ -1 & 1 \end{pmatrix} \\
D &= P^{-1}AP = \frac{1}{2}\begin{pmatrix} 1 & 1 \\ -1 & 1 \end{pmatrix}  \begin{pmatrix}
  1-p & p   \\
  p   & 1-p \\
 \end{pmatrix}\begin{pmatrix} 1 & -1 \\ 1 & 1 \end{pmatrix} = \begin{pmatrix} 1 & 0 \\ 0 & 1 - 2p \end{pmatrix}  \\
A^n &= PD^nP^{-1} =  \begin{pmatrix} 1 & -1 \\ 1 & 1 \end{pmatrix} \begin{pmatrix} 1 & 0 \\ 0 & (1 - 2p)^n \end{pmatrix} \frac{1}{2}\begin{pmatrix} 1 & 1 \\ -1 & 1 \end{pmatrix} \\
& = \begin{pmatrix}
\frac{1}{2}(1 + (1-2p)^n) & \frac{1}{2}(1 - (1-2p)^n) \\\frac{1}{2}(1 - (1-2p)^n) & \frac{1}{2}(1 + (1-2p)^n) \end{pmatrix}  \\
&= \begin{pmatrix} 1 - p_n & p_n \\ p_n & 1 - p_n \end{pmatrix} \textrm{, where } p_n = \frac{1}{2}(1 - (1-2p)^n)
\end{align} 

\end{homeworkProblem} 

\begin{homeworkProblem}[Problem 7.11: Time-varying Channel]
\begin{align}
\max_{p(x^n)} I(X^n;Y^n) &= H(Y^n) - H(Y^n | X^n) \\
\end{align}
\end{homeworkProblem} 

\begin{homeworkProblem}[Problem 7.13: Binary Channel]
\begin{itemize}
\item[a.] 
Let $Pr(X=1) = \pi$, let $E$ be the event $\{Y=e\}$:
\begin{align}
\max_{p(x)} I(X;Y)  &= H(Y) - H(Y|X) \\
H(Y|X) &= H(1-\alpha-\epsilon, \alpha, \epsilon) \\
H(Y) &= H((1-\pi)(1-\alpha-\epsilon) + \pi\epsilon, \alpha, (1-\pi)\epsilon + \pi(1-\alpha-\epsilon)) \\
H(Y) &= H(E) + H(Y|E) \\
H(E) &= H(1-\alpha-\epsilon, \alpha, \epsilon) \\
H(Y|E) &= (1-\alpha)H(\epsilon) \\
\max_{p(x)} I(X;Y)  &= \\
\end{align}
\item[b.] $1 - H(\epsilon)$
\item[c.] $1 - \alpha$
\end{itemize}
\end{homeworkProblem} 

\begin{homeworkProblem}[Problem 7.20: Independent Looks]
\begin{itemize}
\item[a.] 
\begin{align}
I(X;Y_1,Y_2) &= H(Y_1, Y_2) - H(Y_1, Y_2 | X) & \textrm{chain rule} \\
H(Y_1,Y_2|X) &= 2H(Y_1|X) &  \textrm{$Y_i$ is cond. independent on X} \\
&= 2H(Y_1) - 2I(X;Y_1) & \textrm{chain rule} \\
\end{align}
\item[b.]The channel capacity for the single look is $C=\max_{p(x)} I(X;Y_1)$. For $I(X;Y_1,Y_2)=2I(X;Y_1)-I(Y_1,Y_2)$, we note that $p(x)$ for the single look maximizes the double look as well ($I(Y_1,Y_2)$ is independent of the probability distribution of $X$). Therefore, the double look capacity is $2C-I(Y_1, Y_2)$. Since mutual information is always positive, $2C-I(Y_1, Y_2) \leq 2C$.
\end{itemize}
\end{homeworkProblem} 

\end{spacing}
\end{document}
%%%%%%%%%%%%%%%%%%%%%%%%%%%%%%%%%%%%%%%%%%%%%%%%%%%%%%%%%%%%%
