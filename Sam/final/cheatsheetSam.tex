\documentclass[10pt]{article}

\usepackage{fullpage,latexsym,amsthm,amsmath,amsfonts,algorithmic,graphicx,subfigure,color,verbatim,epsfig,fancyhdr}
\usepackage{qtree}
\pagestyle{empty}
% Define new commands.
\DeclareMathOperator*{\E}{\mbox{\text{E}}}

\newcommand{\pref}[1]{{(\ref{#1})}}
\newcommand{\bmat}[1]{{\color{red}{#1}}}

\newtheorem{defn}{Definition}[section]
\newtheorem{thm}{Theorem}[section]
\newtheorem{lemma}{Lemma}[section]
\newtheorem{corollary}{Corollary}[section]
\newtheorem{question}{Question}[section] \newtheorem{proposition}{Proposition}[section] \usepackage{multicol}
\usepackage{savetrees}


% Begin the actual document content here.
\begin{document}
\begin{tiny}
\begin{multicols}{3}

\textit{Huffman Dummy Variables}
Use $1+k(D-1)$ where $D$ is the encoded base, $k$ is the smallest constant such that the expression is $\geq$ the number of variables you are encoding (say $V$). Add $1+k(D-1) - V$ dummy variables.

\textit{5.1: Uniquely Decodable}
$L = \sum_{i=1}^m p_i l_i^{100}, L_1 = min\{L\} \textrm { over instantaneous codes}, L_2 = min\{L\} \textrm{ over uniquely decodable codes}$. We observe that since uniquely decodable codes are a superset of instantaneous codes, we can immediately conclude that $L_2 \leq L_1$, however the inequality can be stricter. Specifically, the Kraft and McMillan inequalities do not depend on the standard expected length definition ($\sum_i p(x_i)l_i$), and apply in the $l_i^{100}$ case as well. Thus, any codeword that minimizes $L_2$ must satisfy McMillan, and thus Kraft. Yet, Kraft also asserts that given a set of lengths that satisfy Kraft, there exists a code. Thus, there exists a code with the same code lengths in $L_1$, and $\fbox{$L_1 = L_2$}$.

\textit{5.1: Martian Fingers}
$(l_1,l_2,...,l_6) = (1,1,2,3,2,3)$ We note that any uniquely decodable D-ary code must satisfy the Kraft inequality: $\sum D^{-l_i} \leq 1$, or, $D^{-1} + D^{-2} + D^{-3} \leq \frac{1}{2}$. Additionally, we assume that the Martian has an integer number of fingers. We know that for $D=2$ the inequality does not hold, however for $D=3$: $\frac{1}{3} + \frac{1}{9} + \frac{1}{27} = \frac{13}{27} \leq  \frac{13}{26} = \frac{1}{2}$. Thus, a Martian has at least $3$ fingers.

\textit{5.4: Huffman Coding}
\[
 X =
 \begin{pmatrix}
	x_1 & x_2 & x_3 & x_4 & x_5 & x_6 & x_7 \\
	0.49 & 0.26 & 0.12 & 0.04 & 0.04 & 0.03 & 0.02 \\
 \end{pmatrix}
\]
\Tree [.$1.0$ [.$0.51$ [.$0.25$ [.$0.13$ [.$0.08$ $x_4$ $x_5$ ] [.$0.05$ $x_6$ $x_7$ ] ] $x_3$ ] $x_2$ ] $x_1$ ]
\begin{itemize}
	\item[a.] 
\[
 \begin{pmatrix}
	x_1 & x_2 & x_3 & x_4   & x_5   & x_6   & x_7 \\
	1   & 01  & 001 & 00001 & 00000 & 00010 & 00011 \\
 \end{pmatrix}
\]
	\item[b.] $L = \sum_{i=1}^7 l_i p(x_i) = 0.04*5 + 0.04*5 + 0.03*5 + 0.02*5 + 0.12*3 + 0.26*2 + 0.49*1 = 2.02\textrm{ bits}$

	\item[c.] 
\
\Tree [.$1.0$  [.$0.25$ [.$0.09$ $x_5$ $x_6$ $x_7$ ] $x_3$ $x_4$ ] $x_2$ $x_1$ ]

\[
 \begin{pmatrix}
	x_1 & x_2 & x_3 & x_4 & x_5 & x_6 & x_7 \\
	0   & 1   & 21  & 20  & 222 & 221 & 220 \\
 \end{pmatrix}
\]
\end{itemize}

\textit{5.6: Bad Codes}

\begin{itemize}
	\item[a.] $ \{0, 10, 11 \} $ This code is a $\fbox{valid}$ Huffman code, which could be generated by the following probabilities $\{ \frac{1}{2}, \frac{1}{4}, \frac{1}{4} \}$, where $D=2$.
	\item[b.] $ \{00, 01, 10, 110 \} $ This code is an $\fbox{invalid}$ Huffman code, as a Huffman code has optimal length, and $110$ can be replaced by $11$ to decrease the expected length.  
	\item[c.] $ \{01, 10 \} $ This code is an $\fbox{invalid}$ Huffman code, as a Huffman code has optimal length and $01$ can be replaced with $0$, while $10$ can be replaced by $1$ to reduce the expected length by $2$.
\end{itemize}

\textit{5.12: Shannon and Huffman}
\begin{itemize}
	\item[a.]
\Tree [.$1.0$ [.$\frac{2}{3}$ [.$\frac{1}{3}$ $\frac{1}{4}$ $\frac{1}{12}$ ] $\frac{1}{3}$ ] $\frac{1}{3}$ ]
\[
 \begin{pmatrix}
	\frac{1}{3} & \frac{1}{3} & \frac{1}{12} & \frac{1}{4}\\
	0           &  10         & 110          & 111\\
 \end{pmatrix}
\]
	\item[b.] We note that the tree above exhibits a $(1,2,3,3)$ code. However, another tree could have been constructed: 

\Tree [.$1.0$ [.$\frac{1}{3}$ $\frac{1}{4}$ $\frac{1}{12}$ ] [.$\frac{2}{3}$ $\frac{1}{3}$ $\frac{1}{3}$ ] ] 

Which has Huffman coding:
\[
 \begin{pmatrix}
	\frac{1}{3} & \frac{1}{3} & \frac{1}{12} & \frac{1}{4}\\
	00           &  01         & 10          & 11\\
 \end{pmatrix}
\] with lengths $(2, 2, 2, 2)$.
	\item[c.] We note that in the first code, the symbol with probability $\frac{1}{4}$ is encoded with $3$ bits, whereas the Shannon code length is $\lceil log(\frac{1}{p(x)}) \rceil = 2$ bits. Thus, there are optimal codes with codeword lengths for some symbols that exceed the Shannon code length.
\end{itemize}

\end{multicols}
\end{tiny}
\end{document}
