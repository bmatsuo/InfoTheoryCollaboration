\documentclass{report}
% Change "report" to "article" to get a page number on title page
\usepackage{fitch}
\usepackage{amsmath,amsfonts,amsthm,amssymb}
\usepackage{setspace}
\usepackage{fancyhdr}
\usepackage{lastpage}
\usepackage{extramarks}
\usepackage{chngpage}
\usepackage{soul,color}
\usepackage{qtree}
\usepackage{graphicx,float,wrapfig}

% Homework Specific Information
\newcommand{\hmwkTitle}{Homework 5}
\newcommand{\hmwkDueDate}{Tues., Nov. 16}
\newcommand{\hmwkClass}{EE253}
\newcommand{\hmwkClassTime}{}
\newcommand{\hmwkClassInstructor}{Prof. Sadjadpour}
\newcommand{\hmwkAuthorName}{Sam Wood}

% In case you need to adjust margins:
\topmargin=-0.45in      %
\evensidemargin=0in     %
\oddsidemargin=0in      %
\textwidth=6.5in        %
\textheight=9.0in       %
\headsep=0.25in         %

% Setup the header and footer
\pagestyle{fancy}                                                       %
\lhead{\hmwkAuthorName}                                                 %
\rhead{\firstxmark}                                                     %
\lfoot{\lastxmark}                                                      %
\cfoot{}                                                                %
\rfoot{Page\ \thepage\ of\ \protect\pageref{LastPage}}                          %
\renewcommand\headrulewidth{0.4pt}                                      %
\renewcommand\footrulewidth{0.4pt}                                      %

% This is used to trace down (pin point) problems
% in latexing a document:
%\tracingall

%%%%%%%%%%%%%%%%%%%%%%%%%%%%%%%%%%%%%%%%%%%%%%%%%%%%%%%%%%%%%
% Some tools
\newcommand{\enterProblemHeader}[1]{\nobreak\extramarks{#1}{#1 continued on next page\ldots}\nobreak%
                                    \nobreak\extramarks{#1 (continued)}{#1 continued on next page\ldots}\nobreak}%
\newcommand{\exitProblemHeader}[1]{\nobreak\extramarks{#1 (continued)}{#1 continued on next page\ldots}\nobreak%
                                   \nobreak\extramarks{#1}{}\nobreak}%

\newlength{\labelLength}
\newcommand{\labelAnswer}[2]
  {\settowidth{\labelLength}{#1}%
   \addtolength{\labelLength}{0.25in}%
   \changetext{}{-\labelLength}{}{}{}%
   \noindent\fbox{\begin{minipage}[c]{\columnwidth}#2\end{minipage}}%
   \marginpar{\fbox{#1}}%

   % We put the blank space above in order to make sure this
   % \marginpar gets correctly placed.
   \changetext{}{+\labelLength}{}{}{}}%

\newcommand{\homeworkProblemName}{}%
\newcounter{homeworkProblemCounter}%
\newenvironment{homeworkProblem}[1][Problem \arabic{homeworkProblemCounter}]%
  {\stepcounter{homeworkProblemCounter}%
   \renewcommand{\homeworkProblemName}{#1}%
   \section*{\homeworkProblemName}%
   \enterProblemHeader{\homeworkProblemName}}%
  {\exitProblemHeader{\homeworkProblemName}}%

\newcommand{\problemAnswer}[1]
  {\noindent\fbox{\begin{minipage}[c]{\columnwidth}#1\end{minipage}}}%

\newcommand{\problemLAnswer}[1]
  {\labelAnswer{\homeworkProblemName}{#1}}

\newcommand{\homeworkSectionName}{}%
\newlength{\homeworkSectionLabelLength}{}%
\newenvironment{homeworkSection}[1]%
  {% We put this space here to make sure we're not connected to the above.
   % Otherwise the changetext can do funny things to the other margin

   \renewcommand{\homeworkSectionName}{#1}%
   \settowidth{\homeworkSectionLabelLength}{\homeworkSectionName}%
   \addtolength{\homeworkSectionLabelLength}{0.25in}%
   \changetext{}{-\homeworkSectionLabelLength}{}{}{}%
   \subsection*{\homeworkSectionName}%
   \enterProblemHeader{\homeworkProblemName\ [\homeworkSectionName]}}%
  {\enterProblemHeader{\homeworkProblemName}%

   % We put the blank space above in order to make sure this margin
   % change doesn't happen too soon (otherwise \sectionAnswer's can
   % get ugly about their \marginpar placement.
   \changetext{}{+\homeworkSectionLabelLength}{}{}{}}%

\newcommand{\sectionAnswer}[1]
  {% We put this space here to make sure we're disconnected from the previous
   % passage

   \noindent\fbox{\begin{minipage}[c]{\columnwidth}#1\end{minipage}}%
   \enterProblemHeader{\homeworkProblemName}\exitProblemHeader{\homeworkProblemName}%
   \marginpar{\fbox{\homeworkSectionName}}%

   % We put the blank space above in order to make sure this
   % \marginpar gets correctly placed.
   }%

%%%%%%%%%%%%%%%%%%%%%%%%%%%%%%%%%%%%%%%%%%%%%%%%%%%%%%%%%%%%%

%%%%%%%%%%%%%%%%%%%%%%%%%%%%%%%%%%%%%%%%%%%%%%%%%%%%%%%%%%%%%
% Make title
\title{\vspace{2in}\textmd{\textbf{\hmwkClass:\ \hmwkTitle}}\\\normalsize\vspace{0.1in}\small{Due\ on\ \hmwkDueDate}\\\vspace{0.1in}\large{\textit{\hmwkClassInstructor\ \hmwkClassTime}}\vspace{3in}}
\date{}
\author{\textbf{\hmwkAuthorName}}
%%%%%%%%%%%%%%%%%%%%%%%%%%%%%%%%%%%%%%%%%%%%%%%%%%%%%%%%%%%%%
\begin{document}
\begin{spacing}{1.1}
\maketitle
\newpage

\begin{homeworkProblem}[Problem 5.1: Uniquely Decodable]
$L = \sum_{i=1}^m p_i l_i^{100}, L_1 = min\{L\} \textrm { over instantaneous codes}, L_2 = min\{L\} \textrm{ over uniquely decodable codes}$. We observe that since uniquely decodable codes are a superset of instantaneous codes, we can immediately conclude that $L_2 \leq L_1$, however the inequality can be stricter. Specifically, the Kraft and McMillan inequalities do not depend on the standard expected length definition ($\sum_i p(x_i)l_i$), and apply in the $l_i^{100}$ case as well. Thus, any codeword that minimizes $L_2$ must satisfy McMillan, and thus Kraft. Yet, Kraft also asserts that given a set of lengths that satisfy Kraft, there exists a code. Thus, there exists a code with the same code lengths in $L_1$, and $\fbox{L_1 = L_2}$.
\end{homeworkProblem} 

\begin{homeworkProblem}[Problem 5.2: Martian Fingers]
$(l_1,l_2,...,l_6) = (1,1,2,3,2,3)$ We note that any uniquely decodable D-ary code must satisfy the Kraft inequality: $\sum D^{-l_i} \leq 1$, or, $D^{-1} + D^{-2} + D^{-3} \leq \frac{1}{2}$. Additionally, we assume that the Martian has an integer number of fingers. We know that for $D=2$ the inequality does not hold, however for $D=3$: $\frac{1}{3} + \frac{1}{9} + \frac{1}{27} = \frac{13}{27} \leq  \frac{13}{26} = \frac{1}{2}$. Thus, a Martian has at least $3$ fingers.
\end{homeworkProblem} 

\begin{homeworkProblem}[Problem 5.4: Huffman Coding]

\[
 X =
 \begin{pmatrix}
	x_1 & x_2 & x_3 & x_4 & x_5 & x_6 & x_7 \\
	0.49 & 0.26 & 0.12 & 0.04 & 0.04 & 0.03 & 0.02 \\
 \end{pmatrix}
\]
\Tree [.$1.0$ [.$0.51$ [.$0.25$ [.$0.13$ [.$0.08$ $x_4$ $x_5$ ] [.$0.05$ $x_6$ $x_7$ ] ] $x_3$ ] $x_2$ ] $x_1$ ]
\begin{itemize}
	\item[a.] 
\[
 \begin{pmatrix}
	x_1 & x_2 & x_3 & x_4   & x_5   & x_6   & x_7 \\
	1   & 01  & 001 & 00001 & 00000 & 00010 & 00011 \\
 \end{pmatrix}
\]
	\item[b.] $L = \sum_{i=1}^7 l_i p(x_i) = 0.04*5 + 0.04*5 + 0.03*5 + 0.02*5 + 0.12*3 + 0.26*2 + 0.49*1 = 2.02\textrm{ bits}$

	\item[c.] 
\
\Tree [.$1.0$  [.$0.25$ [.$0.09$ $x_5$ $x_6$ $x_7$ ] $x_3$ $x_4$ ] $x_2$ $x_1$ ]

\[
 \begin{pmatrix}
	x_1 & x_2 & x_3 & x_4 & x_5 & x_6 & x_7 \\
	0   & 1   & 21  & 20  & 222 & 221 & 220 \\
 \end{pmatrix}
\]
\end{itemize}
\end{homeworkProblem} 

\begin{homeworkProblem}[Problem 5.6: Bad Codes]
\begin{itemize}
	\item[a.] $ \{0, 10, 11 \} $ This code is a $\fbox{valid}$ Huffman code, which could be generated by the following probabilities $\{ \frac{1}{2}, \frac{1}{4}, \frac{1}{4} \}$, where $D=2$.
	\item[b.] $ \{00, 01, 10, 110 \} $ This code is an $\fbox{invalid}$ Huffman code, as a Huffman code has optimal length, and $110$ can be replaced by $11$ to decrease the expected length.  
	\item[c.] $ \{01, 10 \} $ This code is an $\fbox{invalid}$ Huffman code, as a Huffman code has optimal length and $01$ can be replaced with $0$, while $10$ can be replaced by $1$ to reduce the expected length by $2$.
\end{itemize}
\end{homeworkProblem} 

\begin{homeworkProblem}[Problem 5.12: Shannon and Huffman]
\begin{itemize}
	\item[a.]
\Tree [.$1.0$ [.$\frac{2}{3}$ [.$\frac{1}{3}$ $\frac{1}{4}$ $\frac{1}{12}$ ] $\frac{1}{3}$ ] $\frac{1}{3}$ ]
\[
 \begin{pmatrix}
	\frac{1}{3} & \frac{1}{3} & \frac{1}{12} & \frac{1}{4}\\
	0           &  10         & 110          & 111\\
 \end{pmatrix}
\]
	\item[b.] We note that the tree above exhibits a $(1,2,3,3)$ code. However, another tree could have been constructed: 

\Tree [.$1.0$ [.$\frac{1}{3}$ $\frac{1}{4}$ $\frac{1}{12}$ ] [.$\frac{2}{3}$ $\frac{1}{3}$ $\frac{1}{3}$ ] ] 

Which has Huffman coding:
\[
 \begin{pmatrix}
	\frac{1}{3} & \frac{1}{3} & \frac{1}{12} & \frac{1}{4}\\
	00           &  01         & 10          & 11\\
 \end{pmatrix}
\] with lengths $(2, 2, 2, 2)$.
	\item[c.] We note that in the first code, the symbol with probability $\frac{1}{4}$ is encoded with $3$ bits, whereas the Shannon code length is $\lceil log(\frac{1}{p(x)}) \rceil = 2$ bits. Thus, there are optimal codes with codeword lengths for some symbols that exceed the Shannon code length.
\end{itemize}
\end{homeworkProblem} 

\end{spacing}
\end{document}
%%%%%%%%%%%%%%%%%%%%%%%%%%%%%%%%%%%%%%%%%%%%%%%%%%%%%%%%%%%%%
